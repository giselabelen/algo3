\documentclass[a4paper]{article}
\usepackage[spanish]{babel}
\usepackage[utf8]{inputenc}
\usepackage{fancyhdr}
\usepackage{charter}   % tipografía
\usepackage{graphicx}
\usepackage{makeidx}

\usepackage{float}
\usepackage{amsmath, amsthm, amssymb}
\usepackage{amsfonts}
\usepackage{sectsty}
\usepackage{wrapfig}
\usepackage{listings} % necesario para el resaltado de sintaxis
\usepackage{caption}
\usepackage{placeins}

\usepackage{hyperref} % agrega hipervínculos en cada entrada del índice
\hypersetup{          % (en el pdf)
    colorlinks=true,
    linktoc=all,
    citecolor=black,
    filecolor=black,
    linkcolor=black,
    urlcolor=black
}

\usepackage{color} % para snippets de código coloreados
\usepackage{fancybox}  % para el sbox de los snippets de código

\definecolor{litegrey}{gray}{0.94}

% \newenvironment{sidebar}{%
% 	\begin{Sbox}\begin{minipage}{.85\textwidth}}%
% 	{\end{minipage}\end{Sbox}%
% 		\begin{center}\setlength{\fboxsep}{6pt}%
% 		\shadowbox{\TheSbox}\end{center}}
% \newenvironment{warning}{%
% 	\begin{Sbox}\begin{minipage}{.85\textwidth}\sffamily\lite\small\RaggedRight}%
% 	{\end{minipage}\end{Sbox}%
% 		\begin{center}\setlength{\fboxsep}{6pt}%
% 		\colorbox{litegrey}{\TheSbox}\end{center}}

\newenvironment{codesnippet}{%
	\begin{Sbox}\begin{minipage}{\textwidth}\sffamily\small}%
	{\end{minipage}\end{Sbox}%
		\begin{center}%
		\colorbox{litegrey}{\TheSbox}\end{center}}



\usepackage{fancyhdr}
\pagestyle{fancy}

%\renewcommand{\chaptermark}[1]{\markboth{#1}{}}
\renewcommand{\sectionmark}[1]{\markright{\thesection\ - #1}}

\fancyhf{}

\fancyhead[LO]{Sección \rightmark} % \thesection\
\fancyfoot[LO]{\small{integrante 1, integrante 2, integrante 3, ..., integrante n}}
\fancyfoot[RO]{\thepage}
\renewcommand{\headrulewidth}{0.5pt}
\renewcommand{\footrulewidth}{0.5pt}
\setlength{\hoffset}{-0.8in}
\setlength{\textwidth}{16cm}
%\setlength{\hoffset}{-1.1cm}
%\setlength{\textwidth}{16cm}
\setlength{\headsep}{0.5cm}
\setlength{\textheight}{25cm}
\setlength{\voffset}{-0.7in}
\setlength{\headwidth}{\textwidth}
\setlength{\headheight}{13.1pt}

\renewcommand{\baselinestretch}{1.1}  % line spacing


\usepackage{underscore}
\usepackage{caratula}
\usepackage{url}
\usepackage{color}
\usepackage{clrscode3e} % necesario para el pseudocodigo (estilo Cormen)




\begin{document}

\lstset{
  language=C++,                    % (cambiar al lenguaje correspondiente)
  backgroundcolor=\color{white},   % choose the background color
  basicstyle=\footnotesize,        % size of fonts used for the code
  breaklines=true,                 % automatic line breaking only at whitespace
  captionpos=b,                    % sets the caption-position to bottom
  commentstyle=\color{red},    % comment style
  escapeinside={\%*}{*)},          % if you want to add LaTeX within your code
  keywordstyle=\color{blue},       % keyword style
  stringstyle=\color{blue},     % string literal style
}

\thispagestyle{empty}
\materia{Algoritmos y Estructuras de Datos III}
\submateria{Primer Cuatrimestre de 2015}
\titulo{Diseño de Algoritmos}
\subtitulo{Aplicación de técnicas 2.0}
\integrante{Barañao, Facundo}{480/11}{facundo_732@hotmail.com}
\integrante{Confalonieri, Gisela Belén}{511/11}{gise_5291@yahoo.com.ar} % por cada integrante (apellido, nombre) (n° libreta) (e-mail)
\integrante{Mignanelli, Alejandro Rubén}{609/11}{minga_titere@hotmail.com} 

\maketitle
\newpage

\thispagestyle{empty}
\vfill
%\begin{abstract}
%    \vspace{0.5cm}
%	
%
%\end{abstract}

\thispagestyle{empty}
\vspace{1.5cm}
\tableofcontents
\newpage

%\normalsize

\newpage
\section{Objetivos generales}
El objetivo principal del siguiente trabajo será la práctica de distintas técnicas para la creación de algoritmos. Dados tres problemas, que representan situaciones muy factibles de la vida real, se buscará resolverlos de forma óptima mediante la aplicación de algoritmos de programación dinámica y planteamientos del problema en base a grafos. Para cada situación, se buscará que el algoritmo no supere cierta cota de complejidad temporal en peor caso.

%\newpage
\section{Plataforma de pruebas}

Para toda la experimentación se utilizará un procesador Intel Core i3, de 4 nucleos a 2.20 GHZ.

El software utilizado será Ubuntu 14.04, y G++ 4.8.2.
\newpage
\section{Problema 1: Dakkar}
\subsection{Descripción del problema.}

\vspace*{0.3cm}

\textbf{completar!}


%\begin{figure}[htb]
%  \begin{center}
%      \includegraphics[scale=0.25]{imagenes/ejemplo.jpg}
%  \end{center}
%  \caption{ejemplo}
%\end{figure}



\newpage
\subsection{Desarrollo de la idea y pseudocódigo.}

\vspace*{0.3cm}

\textbf{completar!}

\begin{codebox}
\Procname{$\proc{ejemplo_de_pseudocodigo}(x,y)$}
\li \Return $\id{solucion}$
\end{codebox}



\newpage
\subsection{Justificación de la resolución y demostración de correctitud.}

\vspace*{0.3cm}

\textbf{completar!}



\newpage
\subsection{Análisis de complejidad.}

\vspace*{0.3cm}

\textbf{completar!}



\newpage
\subsection{Experimentación y gráficos.}

\vspace*{0.3cm}

\subsubsection{Test 1}

\vspace*{0.3cm}

\textbf{completar!}


\newpage
\subsubsection{Test 2}

\vspace*{0.3cm}

\textbf{completar!}


\newpage
\subsubsection{Test 3}

\vspace*{0.3cm}

\textbf{completar!}


\newpage
\section{Problema 2: Zombieland II}
\subsection{Descripción del problema.}

\vspace*{0.3cm}

Luego del éxito rotundo del último ataque contra los zombies producido por nuestro anterior algoritmo, la humanidad ve un rayo de esperanza.
En una de las últimas ciudades tomadas por la amenaza Z, se encuentra un científico que afirma haber encontrado la cura contra el zombieismo, rodeado por una cuadrilla de soldados del más alto nivel. Nuestro noble objetivo es entonces, proveer del camino más seguro(el que genere menos perdidas humanas) al científico y sus soldados de manera tal que logren llegar desde donde están, hasta el bunker militar que se encuentra en esa ciudad. Para esto, se conocen los siguientes datos:
\begin{itemize}

	\item La ciudad en cuestión tiene la forma clásica de grilla rectangular, compuesta por $n$ calles paralelas en forma horizontal, $m$ calles paralelas en forma vertical dando así una grilla de manzanas cuadradas. Los numeros $n$ y $m$ son conocidos.
	\item Se conoce la ubicación del científico y del bunker.
	\item Se conoce la cantidad de soldados que el científico tiene a su disposición.
	\item Si todos los soldados perecen, el científico no tiene posibilidad de sobrevivir por si mismo(o sea, no puedo llegar al bunker con 0 soldados).
	\item Se sabe la cantidad de zombies ubicados en cada calle.
	\item Si bien nuestros soldados tienen un alto nivel de combate, el enfrentamiento que tuvieron en el último ataque contra los zombies, acabaron con todas sus municiones por lo cual solo tienen sus cuchillos de combate. Esto incide entonces, en que el grupo solo pasara por una calle sin sufrir perdidas, si hay hasta un soldado por zombie, al menos, y en caso contrario, perdera la diferencia entre la cantidad de soldados, y la cantidad de zombies. En otras palabras, sea $z$ la cantidad de zombies de una calle, y $s$ la cantidad de soldados de que tiene la cuadrilla al momento de pasar por esa calle, si $s \geq z$ entonces la cuadrilla pasa sin perdidas. En caso contrario la cuadrilla pierde $z - s$ soldados.
	\item Puede no existir un camino que asegure la supervivencia del científico.
	\item La complejidad del algoritmo pedido es de $\mathcal{O}(s \cdot n \cdot m)$.


\end{itemize}

Ejemplo: 

PENSAR EJEMPLO, Y MOSTRAR LA ENTRADA CON UN DIBUJITO, ES MÁS ENTENDIBLE (ALE)

\vspace*{0.6cm}

%\newpage
\subsection{Desarrollo de la idea y correctitud.}

\vspace*{0.3cm}


\vspace*{0.6cm}

%\newpage
%\subsection{Justificación de la resolución y demostración de correctitud.}

%\vspace*{0.3cm}



%\vspace*{0.6cm}

%\newpage
\subsection{Análisis de complejidad.}

\vspace*{0.3cm}


\vspace*{0.6cm}

%\newpage
\subsection{Experimentación y gráficos.}


\subsubsection{Test 1}

\vspace*{0.3cm}


\vspace*{0.6cm}

%\newpage
\subsubsection{Test 2}


\newpage

\newpage
\section{Problema 3: Refinando petróleo}
\subsection{Descripción del problema.}

\vspace*{0.3cm}

\textbf{completar!}


%\begin{figure}[htb]
%  \begin{center}
%      \includegraphics[scale=0.25]{imagenes/ejemplo.jpg}
%  \end{center}
%  \caption{ejemplo}
%\end{figure}



\newpage
\subsection{Desarrollo de la idea y pseudocódigo.}

\vspace*{0.3cm}

\textbf{completar!}

\begin{codebox}
\Procname{$\proc{ejemplo_de_pseudocodigo}(x,y)$}
\li \Return $\id{solucion}$
\end{codebox}



\newpage
\subsection{Justificación de la resolución y demostración de correctitud.}

\vspace*{0.3cm}

\textbf{completar!}



\newpage
\subsection{Análisis de complejidad.}

\vspace*{0.3cm}

\textbf{completar!}



\newpage
\subsection{Experimentación y gráficos.}

\vspace*{0.3cm}

\subsubsection{Test 1}

\vspace*{0.3cm}

\textbf{completar!}


\newpage
\subsubsection{Test 2}

\vspace*{0.3cm}

\textbf{completar!}


\newpage
\subsubsection{Test 3}

\vspace*{0.3cm}

\textbf{completar!}



\newpage
\section{Apéndice 1: acerca de los tests}
Para reproducir los tests mencionados en el presente informe, se adjunta con la entrega el script {\it tests_nuestros.sh}.  Además, se provee el script {\it compilar.sh} para compilar los programas, y el archivo {\it testing.cpp} que se ocupa de correr los tests para los tres problemas.  Estos dos últimos se llaman desde {\it tests_nuestros.sh}.
El script mencionado se ocupa de realizar las compilaciones necesarias y correr cada una de las pruebas. Los resultados se guardarán en un directorio llamado {\it Resultados_tests_nuestros}.

\newpage
\section{Apéndice 2: secciones relevantes del código}
En esta sección, adjuntamos parte del código correspondiente a la resolución de cada problema que consideramos más \textbf{relevante}.

\subsection{Código del Problema 1}
\begin{lstlisting}
Carrera dakkar(int n, int km, int kb, int* bici, int* moto, int* buggy)
{
	int h;
	int i;
	int j;
	Optimo aux;
	Carrera etapas(n, Etapa(kb+1, Vec(km+1)));
	
	// Lleno los datos de la primera etapa (etapas[0])
	
	// posicion (0,0)
	etapas[0][0][0] = make_pair(bici[0],1);
	
	// posiciones (0,1) a (0,km)
	aux = eleccion(bici[0],moto[0],-1);
	for(j = 1; j <= km; j++){
		etapas[0][0][j] = aux;
	}
	
	// posiciones (1,0) a (kb,0)
	aux = eleccion(bici[0],-1,buggy[0]);
	for(i = 1; i <= kb; i++){
		etapas[0][i][0] = aux;
	}
	
	// demas posiciones
	aux = eleccion(bici[0],moto[0],buggy[0]);
	for(i = 1; i <= kb; i++){
		for(j = 1; j <= km; j++){
			etapas[0][i][j] = aux;
		}
	}
	
	// Armo las demas etapas (etapas[1] a etapas[n-1])
	for(h = 1; h < n; h++)
	{
		// posicion (0,0)
		etapas[h][0][0] = make_pair(etapas[h-1][0][0].first + bici[h],1);
		
		// posiciones (0,1) a (0,km)
		for(j = 1; j <= km; j++){
			aux = eleccion(etapas[h-1][0][j].first + bici[h],etapas[h-1][0][j-1].first + moto[h],-1);
			etapas[h][0][j] = aux;
		}
		
		// posiciones (1,0) a (kb,0)
		for(i = 1; i <= kb; i++){
			aux = eleccion(etapas[h-1][i][0].first + bici[h],-1,etapas[h-1][i-1][0].first + buggy[h]);
			etapas[h][i][0] = aux;
		}
		
		// demas posiciones
		for(i = 1; i <= kb; i++)
		{
			for(j = 1; j <= km; j++)
			{
				aux = eleccion(etapas[h-1][i][j].first + bici[h],etapas[h-1][i][j-1].first + moto[h],etapas[h-1][i-1][j].first + buggy[h]);
				etapas[h][i][j] = aux;
			}
		}
	}
	
	return etapas;
}

\end{lstlisting}

\vspace*{0.5cm}

\begin{lstlisting}
int armo_salida(Carrera etapas, int km, int kb, int n, list<int>& salida)
{
/* Funcion que recorre las matrices de etapas y determina el tiempo
 * total incurrido y el vehiculo utilizado en cada etapa.  Empieza
 * en la ultima etapa y termina con la primera.
 */
	int i = kb;
	int j = km;
	int res = etapas[n-1][i][j].first;	// tiempo total incurrido
	
	for(int h = n-1; h >= 0; h--)
	{
		// apilo vehiculo
		salida.push_front(etapas[h][i][j].second);
		
		// acomodo indices
		if(etapas[h][i][j].second == 2){
			j--;
		}else if(etapas[h][i][j].second == 3){
			i--;
		}
	}
	
	return res;
}

\end{lstlisting}


%\newpage
\subsection{Código del Problema 2}

\begin{lstlisting}
bool recorridos(Mapa& ciudad, list<pair <pos,int> >& cola, int soldados, pos posicion, pos bunker, int& cont, int tope)
{
/* Funcion que recorre el mapa buscando llegar al bunker */
	
	bool res;
	pos pos_aux;
	pair <pos,int> bp;
	int soldAr;
	int soldAb;
	int soldI;
	int soldD;
	
	int i = posicion.horizontal;
	int j = posicion.vertical;
	
	if(posicion == bunker)	// si ya estoy en el bunker, me voy
	{
		return true;
	}
	
	if(ciudad[i][j].derecha != -1)	// si es calle valida
	{
		// si puedo pasar por esa calle
		if((ciudad[i][j].derecha <= soldados) || (2*soldados - ciudad[i][j].derecha >= tope))
		{
			if(ciudad[i][j].derecha <= soldados){
				// si no se muere nadie
				soldD = soldados;
			}else{
				// si se me mueren soldados
				soldD = 2*soldados - ciudad[i][j].derecha;
			}
			
			// aviso que ya pase por esta cuadra
			ciudad[i][j].derecha = -1;
			ciudad[i][j+1].izquierda = -1;
			
			// esquina a la que llego
			pos_aux.horizontal = i;
			pos_aux.vertical = j+1;
			
			if(soldD > ciudad[i][j+1].parcial)	// si vale la pena
			{
				// guardo de donde vine
				ciudad[i][j+1].origen = posicion;
				
				// cantidad de soldados vivos hasta aca
				ciudad[i][j+1].parcial = soldD;
			}
			
			// recursion
			res = recorridos(ciudad, cola, soldD, pos_aux, bunker,cont,tope);
			if(res){ return res; }
		}
	}
	
	if(ciudad[i][j].abajo != -1)	// si es calle valida
	{
		// si puedo pasar por esa calle
		if((ciudad[i][j].abajo <= soldados) || (2*soldados - ciudad[i][j].abajo >= tope))
		{
			if(ciudad[i][j].abajo <= soldados){
				// si no se muere nadie
				soldAb = soldados;
			}else{
				// si se me mueren soldados
				soldAb = 2*soldados - ciudad[i][j].abajo;
			}
				
			// aviso que ya pase por esta cuadra
			ciudad[i][j].abajo = -1;
			ciudad[i+1][j].arriba = -1;
			
			// esquina a la que llego
			pos_aux.horizontal = i+1;
			pos_aux.vertical = j;
			
			if(soldAb > ciudad[i+1][j].parcial)	// si vale la pena
			{
				// guardo de donde vine
				ciudad[i+1][j].origen = posicion;
				
				// cantidad de soldados vivos hasta aca
				ciudad[i+1][j].parcial = soldAb;
			}
			
			// recursion
			res = recorridos(ciudad, cola, soldAb, pos_aux, bunker,cont,tope);
			if(res){ return res; }
		}
	}
	
	if(ciudad[i][j].arriba != -1)	// si es calle valida
	{
		// si puedo pasar por esa calle
		if((ciudad[i][j].arriba <= soldados) || (2*soldados - ciudad[i][j].arriba >= tope))
		{
			if(ciudad[i][j].arriba <= soldados){
				// si no se muere nadie
				soldAr = soldados;
			}else{
				// si se me mueren soldados
				soldAr = 2*soldados - ciudad[i][j].arriba;
			}
						
			// aviso que ya pase por esta cuadra
			ciudad[i][j].arriba = -1;
			ciudad[i-1][j].abajo = -1;
			
			// esquina a la que llego
			pos_aux.horizontal = i-1;
			pos_aux.vertical = j;
			
			if(soldAr > ciudad[i-1][j].parcial)	// si vale la pena
			{
				// guardo de donde vine
				ciudad[i-1][j].origen = posicion;
				
				// cantidad de soldados vivos hasta aca
				ciudad[i-1][j].parcial = soldAr;
			}
			
			// recursion
			res = recorridos(ciudad, cola, soldAr, pos_aux, bunker,cont,tope);
			if(res){ return res; }
		}
	}
	
	if(ciudad[i][j].izquierda != -1)	// si es calle valida
	{
		// si puedo pasar por esa calle
		if((ciudad[i][j].izquierda <= soldados) || (2*soldados - ciudad[i][j].izquierda >= tope))
		{
			if(ciudad[i][j].izquierda <= soldados){
				// si no se muere nadie
				soldI = soldados;
			}else{
				// si se me mueren soldados
				soldI = 2*soldados - ciudad[i][j].izquierda;
			}
		
			// aviso que ya pase por esta cuadra
			ciudad[i][j].izquierda = -1;
			ciudad[i][j-1].derecha = -1;
			
			// esquina a la que llego
			pos_aux.horizontal = i;
			pos_aux.vertical = j-1;
			
			if(soldI > ciudad[i][j-1].parcial)	// si vale la pena
			{
				// guardo de donde vine
				ciudad[i][j-1].origen = posicion;
				
				// cantidad de soldados vivos hasta aca
				ciudad[i][j-1].parcial = soldI;
			}
						
			// recursion
			res = recorridos(ciudad, cola, soldI, pos_aux, bunker,cont,tope);
			if(res){ return res; }
		}
	}
	
	if(ciudad[i][j].arriba > soldados || ciudad[i][j].abajo > soldados ||
		ciudad[i][j].izquierda > soldados || ciudad[i][j].derecha > soldados) 
	{
		bp = make_pair(posicion,soldados);
		cola.push_back(bp);
		cont++;
	}
	
	return false;
}

\end{lstlisting}

\vspace*{0.5cm}

\begin{lstlisting}
void zombieland(Mapa& ciudad, list<pair <pos,int> >& cola, int& soldados, pos bunker)
{
	int contador;
	int cont_aux = 1;
	int sold_aux;
	int tope = soldados;
	bool res = false;
	pos posicion;
	
	while(tope > 0 && !res)	// todavia tengo soldados y no encontre solucion
	{
		contador = cont_aux;	// cant de elementos a mirar en la cola
		cont_aux = 0;
		
		while(contador > 0 && !res)	// todavia tengo elementos para ver y no encontre solucion
		{
			posicion = (cola.front()).first;
			sold_aux = (cola.front()).second;
			cola.pop_front();
			contador--;
	 		res = res || recorridos(ciudad, cola, sold_aux, posicion, bunker, cont_aux, tope);
		}
		
		tope--;
	}
	
	if(!res){ 
		soldados = 0; // se mueren todos
	}
	else{
		soldados = tope + 1; // soldados que llegan vivos
	}

	return;
}

\end{lstlisting}

%\newpage
\subsection{Código del Problema 3}

\begin{lstlisting}

int plan(Incidencia& inc, int n, list<Arista>& tuberias, int& costo_total,vector<int>& padres)
{
/* Funcion que arma el arbol generador minimo con las tuberias y
 * determina las componentes conexas a la vez.  Devuelve la cantidad de
 * tuberias colocadas.
 */
	int costo;
	int p1;
	int p2;
	int res = 0;
	Arista edge;
	vector< pair<int,list<int> > > hijos(n);
	
	for(int i = 0; i < n; i++){	// todos los nodos empiezan con 0 hijos
		hijos[i].first = 0;
	}
	
	for(Incidencia::iterator it1 = inc.begin(); it1 != inc.end(); it1++)
	{
		edge = it1->first;
		costo = it1->second;
		p1 = padres[edge.first];
		p2 = padres[edge.second];
		if((p1 != p2) || (p1 == -1))
		{
			if(p1 != p2)	// padres distintos
			{
				if((p1 != -1) && (p2 != -1))	// si ninguno es -1
				{
					if(hijos[p1].first < hijos[p2].first){
						swap(p1,p2);	// p1 se queda con el que tiene mas hijos
					}
					
					for(list<int>::iterator it2 = (hijos[p2].second).begin(); it2 != (hijos[p2].second).end(); it2++)
					{	// paso los hijos de p2 como hijos de p1
						padres[*it2] = p1;
						(hijos[p1].second).push_back(*it2);
						hijos[p1].first++;
					}
					// pongo a p2 como hijo de p1
					padres[p2] = p1;
					(hijos[p1].second).push_back(p2);
					hijos[p1].first++;
					
				}else if(p1 == -1){	// si p1 es -1
					// pongo a p1 como hijo de p2
					padres[edge.first] = p2;
					(hijos[p2].second).push_back(edge.first);
					hijos[p2].first++;
				}else{	// si p2 es -1
					// pongo a p2 como hijo de p1
					padres[edge.second] = p1;
					(hijos[p1].second).push_back(edge.second);
					hijos[p1].first++;
				}
			}else if (p1 == -1){	// si ambos son -1
				// pongo como padre de ambos a p1
				padres[edge.first] = edge.first;
				padres[edge.second] = edge.first;
				(hijos[edge.first].second).push_back(edge.second);
				hijos[edge.first].first++;
			}
			
			// actualizo variables
			costo_total = costo_total + costo;
			tuberias.push_back(edge);
			res++;
		}
	}
	
	return res;
}
\end{lstlisting}


\vspace*{0.5cm}

\begin{lstlisting}

int petroleo(Incidencia inc,list<int>& refinerias,list<Arista>& tuberias,int n,int& cant_ref,int& cant_tub, int costo_ref)
{
/* Funcion que elabora el plan, determina las refinerias a colocar y 
 * calcula el costo total.  Devuelve el costo total.
 */
	int p;
	int res = 0;
	vector<int> padres(n,-1);
	vector<bool> aux(n,false);
	
	// elaboro el plan de tuberias
	cant_tub = plan(inc,n,tuberias,res,padres);
	
	// determino refinerias
	for(int i = 0; i < n; i++)
	{
		if(padres[i] == -1)	// nodo solitario
		{	// coloco refineria
			refinerias.push_back(i);
			cant_ref++;
		}else{
			p = padres[i];
			if(!aux[p])	// nueva componente conexa
			{	// coloco refineria
				aux[p] = true;
				refinerias.push_back(i);
				cant_ref++;
			}
		}
	}
	
	// actualizo costo total
	res = res + costo_ref*cant_ref;
	return res;
}
\end{lstlisting}


\end{document}

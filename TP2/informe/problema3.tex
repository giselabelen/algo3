\subsection{Descripción del problema.}

\vspace*{0.3cm}

Tenemos en cierta zona, pozos de petróleos que necesitan ser refinados. Para que un pozo pueda refinar su petróleo, necesita o bien una refinería ubicada junto a el, o bien debe conectarse mediante tuberías a otro pozo que o tenga una refinería, o este conectado mediante tuberías a algun pozo que puede refinar su petróleo. Nuestro objetivo es armar un plan de construcción que decida donde construir una refinería y donde construir un sisema de tuberías, de manera tal que todos los pozos puedan refinar su petróleo. Para esto, se tienen los siguientes datos:

\begin{itemize}

	\item Se conoce la cantidad de pozos en cuestión.
	\item Se conoce el costo de construir una refinería(este costo es fijo, y no varía según el pozo).
	\item Dada la geografía del lugar, no todo par de pozos se puede comunicar por una tubería directamente, y por decisiones de administración, una tubería no puede bifurcarse a mitad de camino entre un pozo y otro. Igualmente, conocemos todos los pares de pozos que pueden conectarse con una tubería, y el costo de esta(a diferencia de las refinerías, las tuberías dependen del par de pozos que se quiere conectar).
	\item La complejidad del algoritmo debe ser estrictamente menor a $\mathcal{O}(n^3)$.
	
	
\end{itemize}

\vspace*{0.6cm}

%\newpage
\subsection{Desarrollo de la idea y correctitud.}




%\vspace*{0.6cm}

%\newpage
\subsection{Análisis de complejidad.}

\vspace*{0.3cm}


\vspace*{0.6cm}

%\newpage
\subsection{Experimentación y gráficos.}

\vspace*{0.3cm}


\subsubsection{Test 1}

\vspace*{0.3cm}


\vspace*{0.6cm}

%\newpage
\subsubsection{Test 2}

\vspace*{0.3cm}


\vspace*{0.6cm}

%\newpage
\subsubsection{Test 3}

\vspace*{0.3cm}


\documentclass[a4paper]{article}
\usepackage[spanish]{babel}
\usepackage[utf8]{inputenc}
\usepackage{fancyhdr}
\usepackage{charter}   % tipografía
\usepackage{graphicx}
\usepackage{makeidx}

\usepackage{float}
\usepackage{amsmath, amsthm, amssymb}
\usepackage{amsfonts}
\usepackage{sectsty}
\usepackage{wrapfig}
\usepackage{listings} % necesario para el resaltado de sintaxis
\usepackage{caption}
\usepackage{placeins}

\usepackage{hyperref} % agrega hipervínculos en cada entrada del índice
\hypersetup{          % (en el pdf)
    colorlinks=true,
    linktoc=all,
    citecolor=black,
    filecolor=black,
    linkcolor=black,
    urlcolor=black
}

\input{codesnippet}
\input{page.layout}
\usepackage{underscore}
\usepackage{caratula}
\usepackage{url}
\usepackage{color}
\usepackage{clrscode3e} % necesario para el pseudocodigo (estilo Cormen)




\begin{document}

\lstset{
  language=C++,                    % (cambiar al lenguaje correspondiente)
  backgroundcolor=\color{white},   % choose the background color
  basicstyle=\footnotesize,        % size of fonts used for the code
  breaklines=true,                 % automatic line breaking only at whitespace
  captionpos=b,                    % sets the caption-position to bottom
  commentstyle=\color{red},    % comment style
  escapeinside={\%*}{*)},          % if you want to add LaTeX within your code
  keywordstyle=\color{blue},       % keyword style
  stringstyle=\color{blue},     % string literal style
}

\thispagestyle{empty}
\materia{Algoritmos y Estructuras de Datos III}
\submateria{Primer Cuatrimestre de 2015}
\titulo{Diseño de Algoritmos}
\subtitulo{Aplicación de técnicas}
\integrante{Barañao, Facundo}{480/11}{facundo_732@hotmail.com}
\integrante{Confalonieri, Gisela Belén}{511/11}{gise_5291@yahoo.com.ar} % por cada integrante (apellido, nombre) (n° libreta) (e-mail)
\integrante{Mignanelli, Alejandro Rubén}{609/11}{minga_titere@hotmail.com} 
\integrante{Soliz, Carlos}{406/12}{rcarlos.cs@gmail.com} 

\maketitle
\newpage

\thispagestyle{empty}
\vfill
%\begin{abstract}
%    \vspace{0.5cm}
%	Este trabajo busca aplicar distintas técnicas para la creación de algoritmos. Para esto, se resolverán tres problemas que representan situaciones de la vida real y juegos de ingenio. Las técnicas que se utilizarán serán algoritmos golosos, divide and conquer y backtracking.
%
%\end{abstract}

\thispagestyle{empty}
\vspace{1.5cm}
\tableofcontents
\newpage

%\normalsize

\newpage
\section{Objetivos generales}


%\newpage
\section{Plataforma de pruebas}

Para toda la experimentación se utilizará un procesador Intel Atom, de 4 núcleos a 1.6 GHZ.

El software utilizado será Ubuntu 14.04, y G++ 4.8.2.

\newpage
\section{Problema 1: Dakkar}
\subsection{Descripción del problema.}

\vspace*{0.3cm}

Tenemos un país infectado de zombies. El país esta dividido en ciudades. Nuestro ojetivo es salvar la mayor cantidad de ciudades de este país, enviando soldados a ellas. Una ciudad se considera salvada si los soldados exterminan todos los zombies de dicha ciudad.

Aspectos a tener en cuenta:

\begin{itemize}
   \item Los recursos del país son acotados 
   \item Se conoce la cantidad de soldados atrincherados en cada ciudad (soldados iniciales de la ciudad)  
   \item Se conoce la cantidad de zombies en cada ciudad
   \item Para cada ciudad, se conoce el costo de enviar un soldado
   \item Cada soldado se puede encargar como máximo de 10 zombies 
\end{itemize}

%\begin{figure}[htb]
%  \begin{center}
%      \includegraphics[scale=0.25]{imagenes/ejemplo.jpg}
%  \end{center}
%  \caption{ejemplo}
%\end{figure}
\vspace*{0.6cm}
%\newpage
\subsection{Desarrollo de la idea y pseudocódigo.}

\vspace*{0.3cm}

Notemos que para recuperar la mayor cantidad de ciudades de un país en mano de los zombies, nosotros lo único que podemos hacer es enviar soldados a sus ciudades teniendo en cuenta la cantidad de soldados que esta ciudad posee de por sí. Como esto es la guerra, no tiene ningún sentido enviar soldados a una ciudad sino estamos seguros de que ésta se salvará, puesto que es un desperdicio de presupuesto y vidas humanas. Por otro lado, debido a que queremos salvar la mayor cantidad de ciudades, y sabemos que para salvar una ciudad necesitamos como MÍNIMO 1 soldado por cada diez zombies, si enviamos soldados a una ciudad, nos aseguraremos de enviar la menor cantidad de soldados posibles, así ahorramos recursos que podrán ser utilizados en otra ciudad. Tomando en cuenta todo lo antes dicho, para resolver este problema, la idea será primero tener conocimiento de cuánto cuesta salvar cada ciudad. Para esto, basta hacer la siguiente cuenta por cada ciudad: %(las variables abajo expuestas no pertenecen al algoritmo, son simplemente a modo de ayuda para entender que se hace):
\begin{itemize}
   \item $zombis.vivos = zombis.totales - 10 * soldados.atrincherados.de.la.ciudad$
   \item $soldados.requeridos = \left \lceil \dfrac{zombis.vivos}{10} \right \rceil$ (redondeado hacia arriba, puesto que no puedo enviar medio soldado)
   \item $costo.salvacion.de.la.ciudad = soldados.requeridos * costo.por.soldado.de.la.ciudad$
\end{itemize}

Luego de calcular estos valores, se ordenarán las ciudades de menor a mayor respecto a su costo para ser salvadas y, de manera golosa, irlas salvando en ese orden hasta cubrir el presupuesto.

%\begin{codebox}
%\Procname{$\proc{ejemplo_de_pseudocodigo}(x,y)$}
%\li \Return $\id{solucion}$
%\end{codebox}

\vspace*{0.6cm}

%\newpage
\subsection{Justificación de la resolución y demostración de correctitud.}

\vspace*{0.3cm}

Tenemos dos casos:
\begin{enumerate}
	\item No se salva ninguna ciudad
	\item Se salva al menos una ciudad
\end{enumerate}

\subsubsection{No se salva ninguna ciudad:}

No se salva nadie $\Longleftrightarrow$ nuestro algoritmo dice que no se salva nadie.

$\Longrightarrow )$ Supongamos que no se salva nadie, eso quiere decir que no existe ninguna ciudad tal que si yo gasto todo mi presupuesto en enviarle soldados, ésta obtenga 1 soldado por cada 10 zombies como mínimo.

Sea $p$ un presupuesto tal que no llegue a cumplir los costos de envío de soldados mínimo para salvar ninguna ciudad.
Particularmente, si $p$ no alcanza para salvar a ninguna ciudad, no alcanza para salvar a la ciudad que menos presupuesto necesita para ser salvada (si hay más de una cuyo costo es el mínimo, en particular no se puede salvar ninguna de ellas).$(A)$

Por lo tanto, cuando mi algoritmo ordene las ciudades por costo de salvación, que es justamente el costo que tiene conseguir la mínima cantidad de soldados necesarios para salvar esa ciudad, y tome la primer ciudad, que es la que menos costo tiene para ser salvada (o por lo menos una de ellas), por lo dicho en $(A)$, el algoritmo no tomará esta ciudad ya que el presupuesto no alcanza, y devolverá que se pueden salvar 0 ciudades, y no enviará tropas, el país está perdido, fracasamos, NO SE SALVA NADIE.

$\Longleftarrow )$ Supongamos que nuestro algoritmo dice que no se salva nadie.

Eso significa que por lo explicado anteriormente, el presupuesto es menor a la ciudad que menos costo tiene para ser salvada (si hay varios mínimos, esto no molesta ya que justamente tienen el mismo costo por salvación).

Pero si el presupuesto es menor al costo de salvar a la ciudad (o ciudades) que menos dinero requiere por salvarse, entonces el presupuesto es menor al costo de salvar cualquier ciudad.

Si el presupuesto no alcanza para salvar ninguna ciudad, entonces nuevamente, NO SE SALVA NADIE.

\subsubsection{Se salva al menos una ciudad:}

Hay solución no vacía $\Longleftrightarrow$ nuestro algoritmo devuelve solución óptima.

Sea $N$ nuestra solución óptima, la solución armada con el algoritmo goloso, y $O$ la solución óptima que más se parece a $N$. Asumimos $N$ no óptima y ordenamos $O$ por el costo de salvación, de menor a mayor.
Sea $N_{k}$ el primer elemento en el que $N$ difiere de $O$, o sea $\forall i<k , N_{i}=O_{i}$.

Dado que $N_{k}$ fue obtenido usando el algoritmo goloso sabemos que $Costo(N_{k}) \geq Costo(N{k-1})$ y que $\forall j>k, Costo(N_{k}) \leq Costo(N_{j})$.

En otras palabras, $Costo(N_{k})$ es el menor de todas los costos por salvación mayores a $Costo(N_{k-1})$. Por lo tanto $Costo(N_{k}) \leq Costo(O_{k})$, luego puedo reemplazar $O_{k}$ por $N_{k}$ y obtengo una solución óptima que se parece más que la $O$ original a $N$. ABSURDO puesto que partimos de que $O$ es la solución óptima que más se parece a $N$.

Este absurdo viene de suponer que $N$ no es óptima.  Entonces podemos conluír que, si existe alguna solución, nuestro algoritmo goloso obtiene la solución óptima.

\vspace*{0.6cm}

%\newpage
\subsection{Análisis de complejidad.}

\vspace*{0.3cm}

A continuación mostraremos que la complejidad total de nuestro algoritmo es $\mathcal{O}(n*log(n))$.

Primeramente, veamos la complejidad de la parte golosa (Figura \ref{code:goloso}).  Las primeras dos líneas son asignaciones, así que cuentan como $\mathcal{O}(1)$. En la línea 3 comienza un ciclo que recorrerá las ciudades (previamente ordenadas por costo de salvación) hasta que se supere el presupuesto o se terminen de ver las ciudades. Dentro de ese ciclo, se va acumulando el costo de salvar la ciudad, y en caso de estar dentro del presupuesto, se la salva.  La línea 6 será sólo una asignación a una variable que determinará que dicha ciudad será salvada, así que como sólo tenemos una comparación, un incremento y a lo sumo dos asignaciones, cada iteración del ciclo tendrá una complejidad de $\mathcal{O}(1)$.  Como el ciclo itera a lo sumo n veces, podemos decir que esta parte tiene complejidad $\mathcal{O}(n)$.

\begin{figure}[!ht]
\begin{codebox}
\Procname{int $\proc{zombie_goloso}(lista$_$ciudades, int$ $n, int$ $p)$}
\li i $\leftarrow$ 0
\li sum $\leftarrow$ 0
\li \While ($i<n$ y $sum<p$)    
\li 	\quad	$sum$ + costo_salvar_ciudad_i   
\li	\quad	\If $sum<p$
\li	\quad\quad	salvar_ciudad_i
\li	\quad	$i++$
\li \Return $\id{i}$       
\end{codebox} 
\caption{Pseudocódigo del algoritmo goloso}\label{code:goloso}
\end{figure}
\FloatBarrier

Analicemos ahora la complejidad del algoritmo completo (Figura \ref{code:zombieland}). Las primeras dos líneas son lecturas de datos, las cuales pueden considerarse $\mathcal{O}(1)$.  En la línea 3 se crea una lista con las n ciudades indicadas; si consideramos $\mathcal{O}(1)$ la creación de una lista vacía y $\mathcal{O}(1)$ la lectura y escritura de cada una de las ciudades, podemos decir que este paso es $\mathcal{O}(n)$.  La línea 4 calcula el costo de salvar cada ciudad, para lo cual se recorre toda la lista de ciudades y se hacen los cálculos previamente descritos, así que nuevamente tenemos una complejidad de $\mathcal{O}(n)$.  Luego, en la línea 5, pasamos a ordenar las ciudades por ese costo de salvación, y sabemos que este ordenamiento puede ser implementado con una complejidad de $\mathcal{O}(n*log(n))$. Acto seguido, se hace el llamado a la función golosa analizada anteriormente ($\mathcal{O}(n)$) y se vuelve a ordenar la lista, esta vez por el nombre (un número) que identifica a cada ciudad según el orden en el que vino dada en la entrada (nuevamente, esto puede realizarse en $\mathcal{O}(n*log(n))$). Optamos por este último ordenamiento para facilitar la devolución en el formato indicado por el ejercicio.  Lo último que se hace es sacar por pantalla el número de ciudades salvadas y los soldados enviados a cada ciudad, para lo cual se recorre debe recorrer toda la lista de ciudades ($\mathcal{O}(n)$).

\begin{figure}[!ht]
\begin{codebox}
\Procname{$\proc{Zombieland}$}
\li n $\leftarrow$ Cantidad de ciudades
\li p $\leftarrow$ Presupuesto
\li cities $\leftarrow$ Lista con las n ciudades
\li {\it calcular_costo_de_salvacion}(cities)
\li {\it ordenar_por_costo}(cities)
\li ciudades_salvadas = {\it zombie_goloso}(cities,n,p)
\li {\it ordenar_por_nombre}(cities)
\li Mostrar por pantalla: ciudades salvadas y soldados enviados a cada ciudad
\end{codebox}
\caption{Pseudocódigo de Zombieland}\label{code:zombieland}
\end{figure}
\FloatBarrier

Tenemos entonces la siguiente ecuación:

\begin{equation*}
\begin{array}{l}
T(n) = \mathcal{O}(1) + 4\mathcal{O}(n) + 2\mathcal{O}(n*log(n))\\
T(n) = \mathcal{O}(n*log(n))
\end{array}
\end{equation*}

La complejidad total de este algoritmo es, entonces, $\mathcal{O}(n*log(n))$.


\newpage
\subsection{Experimentación y gráficos.}

\vspace*{0.3cm}

\subsubsection{Test 1}

\vspace*{0.3cm}

\textbf{completar!}


\newpage
\subsubsection{Test 2}

\vspace*{0.3cm}

\textbf{completar!}


\newpage
\subsubsection{Test 3}

\vspace*{0.3cm}

\textbf{completar!}


\newpage
\section{Problema 2: Zombieland II}
\subsection{Descripción del problema.}

\vspace*{0.3cm}

\textbf{
Tenemos un enlace por el cual podemos transmitir información, utilizando distintas frecuencias.
Nuestro objetivo es transmitir información durante todo el tiempo que sea posible, pero invirtiendo la menor cantidad de dinero posible.
\\
Hay que tener en cuenta lo siguiente:
}
\begin{itemize}
	\item vamos a transmitir información secuencialmente, es decir solamente se utilizara una frecuencia por intervalo tiempo 
	\item cada frecuencia tiene un intervalos $<$inicio, fin$>$ en el cual puede ser utilizado y un costo por minuto de uso.Tanto inicio como fin son numeros enteros que representan minutos. 
	\item El enlace puede eligir transmitir por cualquiera de las frecuencias disponibles, y cambiar de una frecuencia a otra.
	\item Se puede cambiar la frecuencia usada a lo largo del tiempo, pero solo una vez por minuto y al comienzo del mismo.
\end{itemize}

\textbf{Nuestro algoritmo toma:}
\begin{itemize}
	\item n que indica la cantidad frecuencias que puede utilizar el enlace
	\item tenemos n lineas seguidas, cada linea indica una frecuencia
	\begin{itemize}
		\item cada linea tiene campos $<$c,i,f$>$ que indican costo por minuto, inicio y fin del intervalo de disponibilidad
	\end{itemize}
	\item El tiempo inicial empieza a partir del minuto cero
\end{itemize}

\textbf{Salida:}
\begin{itemize}
	  \item costo total incurrido de la solucion( osea minimizando los costos)
	  \item una linea por cada intervalo de transmicion 
\end{itemize}

%\begin{figure}[htb]
%  \begin{center}
%      \includegraphics[scale=0.25]{imagenes/ejemplo.jpg}
%  \end{center}
%  \caption{ejemplo}
%\end{figure}



\newpage
\subsection{Desarrollo de la idea y pseudocódigo.}

\vspace*{0.3cm}

\textbf{completar!}

\newpage
\subsection{Justificación de la resolución y demostración de correctitud.}

\vspace*{0.3cm}

\textbf{completar!}



\newpage
\subsection{Análisis de complejidad.}

\vspace*{0.3cm}

En este apartado demostraremos que la complejidad total de nuestro algoritmo es $\mathcal{O}(n*log(n))$.

Empezaremos analizando el pseudocódigo de AltaFrecuencia (Figura \ref{code:altafrec}) asumiendo que la complejidad de frecuency_dc es $\mathcal{O}(n*log(n))$
En las dos primeras lineas se procede a crear una lista vacía $\mathcal{O}(1)$ y una lista que se completará con las frecuencias tomadas de la entrada. Esto último requiere recorrer la entrada $\mathcal{O}(n)$ y copiar los datos $\mathcal{O}(1)$.
En la línea 3 de la figura se pasa a ordenar la lista en base a los tiempos de inicio de cada elemento $\mathcal{O}(n*log(n))$ para luego en la línea 4 ser trasladada a un arreglo previamente creado $\mathcal{O}(n)$. Con dicho arreglo y la función frecuency_dc se arma lo que será la lista solución del problema $\mathcal{O}(n*log(n))$. A continuación, se recorre la lista solución calculando el costo total de la transmisión $\mathcal{O}(n)$ para luego, en la línea 8 mostrar ese valor junto con las frecuencias utilizadas. Esto último se considera $\mathcal{O}(n)$ ya que los costos de mostrar datos son tomados como $\mathcal{O}(1)$ y el costo de recorrer la solución es $\mathcal{O}(n)$.

\begin{figure}[!ht]
\begin{codebox}
\Procname{$\proc{AltaFrecuencia}()$} 
\li Creo una lista de transmisiones resultado  //$\mathcal{O}(1)$ 
\li freq_l $\leftarrow$ Creo una lista para almacenar las n frecuencias levantadas de la entrada
\li ordeno_por_tiempo(freq_l) {\it //ordeno por tiempo de inicio. $\mathcal{O}(n.logn)$}
\li total_freq $\leftarrow$ paso freq_l ordenada a un arreglo $\mathcal{O}(n)$
\li resultado $\leftarrow$ frecuency_dc(total_freq,0,n-1)
\li {\it //armo la transmisión resultado haciendo uso de la técnica de D\&C}
\li costo_total $\leftarrow$ costo_transmision(resultado) //$\mathcal{O}(n)$
\li Muestro por pantalla: costo de la transmisión (costo_total) y el intervalo de tiempo ocupado por cada\\ frecuencia usada {\it //Se hace en $\mathcal{O}(n)$ porque es recorrer la lista mostrando las frecuencias $\mathcal{O}(1)$}
\end{codebox}
\caption{Pseudocódigo de AltaFrecuencia}\label{code:altafrec}
\end{figure}
\FloatBarrier

Luego con la siguiente ecuación:

\begin{equation*}
\begin{array}{l}
T(n) = \mathcal{O}(1) + 2\mathcal{O}(n)*\mathcal{O}(1) + 2\mathcal{O}(n*log(n)) + 3\mathcal{O}(n)\\
T(n) = \mathcal{O}(n*log(n))
\end{array}
\end{equation*}

La complejidad total de este algoritmo es, entonces, $\mathcal{O}(n*log(n))$.

Resta ver que frecuency_dc es en efecto $\mathcal{O}(n*log(n))$.

\begin{codebox}
\Procname{$\proc{frecuency_dc}(Arreglo$_$frecuencias$ $freq, int$ $low, int$ $high)$}
\li Inicializo 3 listas: trans1,trans2,trans_final
\li \If ($low<high$)
\li	\quad	mitad $\longleftarrow$ $\frac{low + high}{2}$
\li	\quad	trans1 $\longleftarrow$ $frequency$_$dc(freq,low,mitad)$
\li	\quad	trans2 $\longleftarrow$ $frequency$_$dc(freq,mitad+1,high)$
\li	\quad	A continuación combino ambas listas de transmisiones de manera que queden ordenadas por costo
\\ \quad de menor a mayor y que las frecuencias alternen de ser necesario.
\li \Else
\li \quad	Armo una lista cuyo único elemento es la frecuencia de la posicion "low". Libero la memoria ocupada\\ \quad por trans1 y trans2
\li \Return trans_final 
 \end{codebox}


\newpage
\subsection{Experimentación y gráficos.}

\vspace*{0.3cm}

\subsubsection{Test 1}

\vspace*{0.3cm}

\textbf{completar!}


\newpage
\subsubsection{Test 2}

\vspace*{0.3cm}

\textbf{completar!}


\newpage
\subsubsection{Test 3}

\vspace*{0.3cm}

\textbf{completar!}


\newpage
\section{Problema 3: Refinando petróleo}
\subsection{Descripción del problema.}

\vspace*{0.3cm}

Se tiene un juego de mesa cuyo tablero, dividido en casillas, posee igual cantidad de filas y columnas y hace uso de una conocida pieza del popular ajedrez: el caballo. El juego es solamente para un jugador y consiste en, teniendo caballos ubicados en distintos casilleros, insertar en casilleros vacíos la mínima cantidad de caballos extras, de manera tal que, siguendo las reglas del movimiento de los caballos en el ajedrez, todas las casillas se encuentren ocupadas o amenazadas por un caballo.
Aspectos a tener en cuenta:

\begin{itemize}
   \item Se conoce la cantidad de filas y columnas del tablero
   \item Se conoce la cantidad de caballos que ocupan el tablero inicialmente.
   \item Para cada uno de estos caballos, se sabe su ubicación en el tablero
   \item Una casilla se considera amenazada si existe un caballo tal que en una movida pueda ocupar dicha casilla
\end{itemize}

%\begin{figure}[htb]
%  \begin{center}
%      \includegraphics[scale=0.25]{imagenes/ejemplo.jpg}
%  \end{center}
%  \caption{ejemplo}
%\end{figure}

\vspace*{0.6cm}

%\newpage
\subsection{Desarrollo de la idea y correctitud.}


\begin{figure}[!htb]
\minipage{0.32\textwidth}
  \includegraphics[scale=1]{imagenes/tab1.png}
  \caption{Tablero 1}\label{fig:tab1}
\endminipage\hfill
\minipage{0.32\textwidth}
  \includegraphics[scale=1]{imagenes/tab2.png}
  \caption{Tablero 2}\label{fig:tab2}
\endminipage\hfill
\minipage{0.32\textwidth}%
  \includegraphics[scale=1]{imagenes/tab3.png}
  \caption{Tablero 3}\label{fig:tab3}
\endminipage
\end{figure}

%\begin{figure}[htb]
 % \begin{center}
%     \includegraphics[scale=1]{imagenes/tab1.png}
%  \end{center}
 % \caption{Tablero 1}\label{fig:tab1}
%\end{figure}
%\vspace*{0.3cm}

%\begin{figure}[htb]
 % \begin{center}
  %    \includegraphics[scale=1]{imagenes/tab2.png}
  %\end{center}
  %\caption{Tablero 2}\label{fig:tab2}
%\end{figure}

%\begin{figure}[htb]
 % \begin{center}
  %    \includegraphics[scale=1]{imagenes/tab3.png}
  %\end{center}
  %\caption{Tablero 3}\label{fig:tab3}
%\end{figure}

Supongamos un ejemplo como el del Figura \ref{fig:tab1}, donde 'v' indica casillero vacío y 'p' un caballo preubicado. Observemos que tanto Figura \ref{fig:tab2} y Figura \ref{fig:tab3} son soluciones (donde 'e' indica que se agregó un caballo extra y 'a' que dicha casilla esta amenazada sin un caballo colocado). Esto significa que en principio, puede haber múltiples soluciones óptimas. Más aún, veamos que si el tablero fuera de 2x2, la única manera de obtener un tablero completo (diremos que un tablero está completo si todas sus casillas, o tienen un caballo, o están amenazadas por uno) es que haya un caballo en cada casilla, o sea que puedo tener soluciones óptimas como en 3x3 o puedo necesitar completar el tablero con caballos como en 2x2. 

Veamos qué sucede para 4x4.


\begin{figure}[!htb]
\minipage{0.32\textwidth}
  \includegraphics[scale=1]{imagenes/tab4.png}
  \caption{Tablero 4}\label{fig:tab4}
\endminipage\hfill
\minipage{0.32\textwidth}
  \includegraphics[scale=1]{imagenes/tab5.png}
  \caption{Tablero 5}\label{fig:tab5}
\endminipage\hfill
\minipage{0.32\textwidth}%
  \includegraphics[scale=1]{imagenes/tab6.png}
  \caption{Tablero 6}\label{fig:tab6}
\endminipage
\end{figure}

%\begin{figure}[htb]
% \begin{center}
%    \includegraphics[scale=1]{imagenes/tab4.png}
%  \end{center}
%  \caption{Tablero 4}\label{fig:tab4}
%\end{figure}

%\begin{figure}[htb]
%  \begin{center}
%      \includegraphics[scale=1]{imagenes/tab5.png}
%  \end{center}
%  \caption{Tablero 5}\label{fig:tab5}
%\end{figure}

%\begin{figure}[htb]
%  \begin{center}
%      \includegraphics[scale=1]{imagenes/tab6.png}
%  \end{center}
%  \caption{Tablero 6}\label{fig:tab6}
%\end{figure}

\begin{figure}[!htb]
\minipage{0.32\textwidth}
  \includegraphics[scale=1]{imagenes/tab7.png}
  \caption{Tablero 7}\label{fig:tab7}
\endminipage
\minipage{0.32\textwidth}
  \includegraphics[scale=1]{imagenes/tab8.png}
  \caption{Tablero 8}\label{fig:tab8}
\endminipage
\end{figure}

%\begin{figure}[htb]
%  \begin{center}
%      \includegraphics[scale=1]{imagenes/tab7.png}
%  \end{center}
%  \caption{Tablero 7}\label{fig:tab7}
%\end{figure}

%\begin{figure}[htb]
%  \begin{center}
%      \includegraphics[scale=1]{imagenes/tab8.png}
%  \end{center}
%  \caption{Tablero 8}\label{fig:tab8}
%\end{figure}

Notemos que la Figura \ref{fig:tab4} es un ejemplo en el que pudimos insertar 2 caballos, que tanto la Figura \ref{fig:tab5} como la Figura \ref{fig:tab6} son solución de un mismo caso inicial (el tablero que tiene solo los caballos preubicados), y que en el Figura \ref{fig:tab7} no es necesario agregar ningún caballo extra. Esto significa que hay casos muy variados y amplios, lo cual vuelve difícil construir un algoritmo que resuelva el problema. Ante esto, como somos malos perdedores, decidimos sacrificar tiempo, pero vamos a encontrar una solución óptima.

Podemos afrontar este problema aplicando la técnica de backtracking, planteando un algoritmo que chequee todas las posibles formas de insertar caballos en un determinado tablero, desde rellenar un tablero con caballos hasta no poner ninguno, y se quede con una solución tal que el tablero esté completo y la cantidad de caballos insertados sea menor o igual al de todo el resto de las soluciones.

Una idea más ordenada de este algoritmo de backtracking es la siguiente (suponiendo que el tablero de entrada no está completo, puesto que sino se devolverá este mismo tablero):
\begin{itemize}
\item Recorriendo el tablero: empezamos por el casillero de la primera fila y columna para ir avanzando por los distintos casilleros de esa fila, en orden, y cuando se llegue al final de una fila, continuaremos por el primer casillero de la fila siguiente. De esta manera, recorremos el tablero completo, asegurandonos que se pasa por todas las casillas (A).
\item Insertando caballos: para cada casillero, si este tiene un caballo "p" colocado, se verá el siguiente casillero, y de no existir, la solución dada hasta el momento se tomará como la óptima. Sino, abordaremos dos posibilidades. Primero veremos el caso en el que se coloca un caballo en ese casillero, y después veremos el caso en el que no se agrega dicho caballo \ref{fig:diag}. En cada caso, se verá si el tablero queda o no completo, y si queda completo, se lo comparará con el resto de las soluciones completas, para ver si la cantidad de caballos agregados es menor o igual a todas o no. En caso de serlo, se la considerará la solución hasta que se encuentre otra mejor. De no encontrarse una solución mejor en los siguientes casos que se observen, será considerada mi solución óptima. Junto con (A), esto significa que se ven todas las posibles ramas de decisión \ref{fig:diag}, y por cada decisión se pregunta si es una solución, y si lo es, si es mejor que las anteriores, lo que quiere decir que pregunta por todas las soluciones y se queda con la mejor efectivamente.
\end{itemize}
\begin{figure}[htb]
  \begin{center}
      \includegraphics[scale=0.6]{imagenes/tab9.png}
  \end{center}
  \caption{Diagrama de acción}\label{fig:diag}
\end{figure}

Teniendo en cuenta estos métodos, llegamos a la conclusión de que nuestro algoritmo es correcto, sin embargo, ver todos los casos resulta tedioso y si bien tenemos tiempo, los docentes de algo 3 se molestarían si perdemos demasiado, lo cual motiva a aplicar cotas estratégicas. Estas funcionarán de la siguiente manera, si un caso dado supera a la cota, se desechará, puesto que ese camino seguro no llega a una solución óptima. De encontrarse una solución que mejora a la cota, entonces esta solución pasará a ser la nueva cota.

Analicemos qué cotas tomar:

No podemos pensar un caso por cada tamaño de tablero, puesto que no terminaríamos nunca, pero sí podemos pensar que si vemos el Tablero de la Figura \ref{fig:tab1} y reemplazamos la 'p' por una 'v', la solución Tablero \ref{fig:tab2} es óptima (cambiando su 'p' por una 'e') lo cual significa que, independientemente de los caballos 'p', con a lo sumo 4 se debe completar un tablero. Por otro lado, como mencionamos previamente, a un tablero de 2x2 hay que llenarlo de caballos, y si vemos el Tablero de la Figura \ref{fig:tab8}, vemos que a un tablero de 4x4 vacío, con 4 caballos le basta para completarse, lo cual indica que sucede algo similar con el tablero de 3x3. Por esto diremos que si nuestro tablero es menor o igual a 4x4, no debemos ver aquellos casos que pongan más de 4 caballos, puesto que de seguro, eso no es una solución óptima. ¿Pero qué sucede con el resto de los casos?. Observemos lo siguiente, sea un tablero de nxm (no apto para este juego, pero sí para lo que se quiere mostrar) con n=5 y m$\geq$5. Tomemos por ejemplo un tablero de 5x5 vacío, pero con una fila de caballos en el centro (llamados c).


\begin{figure}[!htb]
\minipage{0.32\textwidth}
  \includegraphics[scale=1]{imagenes/tab10.png}
  \caption{Tablero 10}\label{fig:tab10}
\endminipage
\minipage{0.32\textwidth}
  \includegraphics[scale=1]{imagenes/tab11.png}
  \caption{Tablero 11}\label{fig:tab11}
\endminipage
\end{figure}

%\begin{figure}[htb]
%  \begin{center}
%      \includegraphics[scale=1]{imagenes/tab10.png}
%  \end{center}
%  \caption{Tablero 9}\label{fig:tab9}
%\end{figure}

%\begin{figure}[htb]
%  \begin{center}
%      \includegraphics[scale=1]{imagenes/tab11.png}
%  \end{center}
%  \caption{Tablero 10}\label{fig:tab10}
%\end{figure}

Notemos que en la Figura \ref{fig:diag} todas las casillas están amenazadas u ocupadas y sería un tablero completo. Es más, si sacáramos un solo caballo, el tablero dejaría de ser completo. Esto, si bien no tiene por que ser una solución óptima, es cuanto menos una solución que no usa una cantidad abusiva de caballos. ¿Qué sucedería si extendiéramos una columna más, como en Figura \ref{fig:tab10}?. Todas las casillas estarían ocupadas menos la de la fila del medio a la derecha. Osea que si la fila estuviera completa, sería un tablero completo. Nótese que si vuelvo a extender una columna sucede lo mismo. Volviendo otra vez a tableros válidos (o sea, de nxn), la idea recién mencionada nos hace pensar que, independientemente del tamaño del tablero, con una línea de caballos cada 5 filas, debería bastar para completar el tablero, y ver más caballos que eso sería un esfuerzo innecesario. Si el tablero no fuera múltiplo de 5, bastaría con poner 1 fila de caballos cada 5 filas normales, y de lo que me queda elijo una fila que sea céntrica respecto de las sobrantes, que son menores que 5, y la convierto en fila de caballos. Ver que si una fila céntrica completa 5 filas, en particular completa 4,3,2 y 1.

Por todo lo dicho anteriormente, si llamamos n a la cantidad de filas y columnas de un tablero, dado un tablero de nxn con n$<$5, tomamos como cota 4 caballos extras, y si n$\geq5$ tomo como cota $n*\left \lceil \dfrac{n}{5} \right \rceil$, que resultaría de meter una fila de caballos por cada 5 filas del tablero.

Sin embargo, agregar estas cotas, ¿puede producir que se obtenga un resultado no óptimo?. La pregunta casi que se contesta sola, pues las cotas justamente se basan en que conocemos una solución segura que, aunque no sabemos si es óptima, sí sabemoms que usa una determinada cantidad de caballos, y una solución que use más caballos que esta seguro no es óptima.

\begin{figure}
\begin{codebox}
\Procname{$\proc{El_señor_de_los_caballos}$} 
\li n $\leftarrow$ tamaño del tablero
\li k $\leftarrow$ cantidad de caballos preubicados
\li extras $\leftarrow$ 0
\li tab $\leftarrow$ tablero de  nxn ``vacío''
\li tab_final $\leftarrow$ tablero de nxn ``vacío'' donde irá el resultado
\li tab $\leftarrow$ setear ubicación de los caballos preubicados y las casillas que amenaza cada uno.
\li falta_cubrir $\leftarrow$ cantidad de casillas no ocupadas 
\li \If ($n < 5$)
\li \Then cota $\leftarrow$ 4
\li \Else cota $\leftarrow n*\left \lceil \dfrac{n}{5} \right \rceil$
	\End
\li {\sc Backtracking}(tab,tab_final,0,0,n)
\li Muestro por pantalla: cantidad de caballos agregados y la posición de cada caballo extra
\end{codebox}
\caption{Pseudocódigo de El señor de los caballos}\label{code:caballos}
\end{figure}
\FloatBarrier

\begin{figure}[!ht]
\begin{codebox}
\Procname{$\proc{Backtracking}(Tablero$ $tab, Tablero$ $tab$_$final, int$ $fila, int$ $columna, int$ $n)$}
\li \If agregué más caballos de los que permitía la cota
\li \Then \Return   
	\End
\li \If ya llené el tablero
\li \Then tab_final $\leftarrow$ copia de tab
\li 		 cota $\leftarrow$ cantidad de caballos agregados 
\li 		\Return
	\End
\li \If ya recorrí todo el tablero
\li \Then \Return
	\End
\li f $\leftarrow$ fila para la siguiente llamada recursiva
\li c $\leftarrow$ columna para la siguiente llamada recursiva 
\li \If en tab[fila][columna] no había un caballo preubicado
\li \Then copia_tab $\leftarrow$ copia de tab
\li 		 r $\leftarrow$ cantidad de casillas que falta llenar
\li		 e $\leftarrow$ cantidad de caballos agregados hasta ahora
\li 		\If copia_tab[fila][columna] no está cubierta
\li 		\Then decrementar r
		\End
\li 		copia_tab[fila][columna] $\leftarrow$ coloco un caballo extra en esa posición
\li		copia_tab $\leftarrow$ seteo las casillas que amenaza ese nuevo caballo extra
\li 		incrementar e
\li 		{\sc Backtraking}(copia_tab,tab_final,f,c,n)
		\End
\li {\sc Backtraking}(tab,tab_final,f,c,n)
\end{codebox}
\caption{Pseudocódigo del backtraking}\label{code:backtraking}
\end{figure}
%\FloatBarrier

%\textbf{completar!}

%\begin{codebox}
%\Procname{$\proc{ejemplo_de_pseudocodigo}(x,y)$}
%\li \Return $\id{solucion}$
%\end{codebox}

\vspace*{0.6cm}

%\newpage
%\subsection{Justificación de la resolución y demostración de correctitud.}

%\vspace*{0.3cm}

%\textbf{completar!}

%\vspace*{0.6cm}

%\newpage
\subsection{Análisis de complejidad.}

\vspace*{0.3cm}

Ahora procederemos a demostrar que el algoritmo que planteamos posee una complejidad de $\mathcal{O}(n^2*2^{n^2})$. Luego de un simple vistazo al pseudocodigo (Figura \ref{code:caballos}) se puede ver que la complejidad deriva en gran medida de lo que sucede cuando se llama a la función encargada del backtracking, dado que las comparaciones realizadas son $\mathcal{O}(1)$ asi como las asignaciones. La complejidad de copiar la entrada y generar el tablero adecuado nos resulta $\mathcal{O}(n^2)$.

Analicemos ahora un caso particular, como el de un tablero de 2x2 sin caballos preubicados. Haciendo un leve seguimiento del pseudocodigo (Figura \ref{code:backtraking}) nos encontramos revisando la primer casilla del tablero y dado que no hay un caballo preubicado se generan 2 instancias, un nuevo tablero con un caballo extra en esa posicion y otro en el que este no se encuentra presente. La creación de este nuevo tablero nos cuesta $\mathcal{O}(n^2)$. Sobre estas dos instancias se vuelve a llamar a la función de backtracking y el proceso se repite ahora con la casilla siguiente, y asi hasta cubrir las $n*n$ casillas. Al terminar el recorrido notamos que contamos con 15 instancias ($2^{n^2} - 1$) y que por cada una tenemos un costo de $\mathcal{O}(n^2)$.
Este proceso de recorrer todas las casillas se repite para cualquier tablero vacío, creando $2^n -1$ instancias y llegando a una complejidad de $\mathcal{O}(n^2*2^{n^2})$ (el -1 es despreciable).

Ahora lo que sucede cuando contamos con caballos preubicados en un tablero cualquiera es que la comparación realizada en la linea 12 de Figura Backtracking, nos ahorra la creacion y copia del nuevo tablero, restando un $\mathcal{O}(n^2)$ a nuestra complejidad. A su vez, la cantidad de ramas con posibles soluciones se ven reducidas considerablemente resultando en $2^{n^2-k}$ instancias.

En conclusión nuestro algoritmo, en el peor caso, dado que no hay caballos prefijados, tiene la complejidad de $\mathcal{O}(n^2*2^{n^2})$.

\vspace*{0.6cm}

%\newpage
\subsection{Experimentación y gráficos.}

\vspace*{0.3cm}

En esta sección normalmente trataríamos de analizar empíricamente que la complejidad de nuestro algoritmo es exponencial. Sin embargo, si realmente fuera exponencial, entonces hacer un análisis de complejidad sería muy complicado por un tema de tiempos. Es por esto, que los experimentos realizados, mostrarán cómo al aumentar el valor de n, los tiempos cambian drásticamente. Por otro lado, veremos cómo el agregar caballos preubicados disminuye en gran proporción el tiempo utilizado por nuestro algoritmo, y qué ocurre con el caso de un tablero lleno.

\subsubsection{Test 1}

\vspace*{0.3cm}

Como primera medida, creamos instancias de tableros de 2x2, 3x3, 4x4 y 5x5, todos vacíos, es decir, sin caballos preubicados.  Cada instancia se ejecutó 50 veces y se consideró el promedio de los tiempos incurridos para cada caso (Figura \ref{fig:3-vacios}).

\begin{figure}[htb]
	\begin{center}
    		\includegraphics[scale=0.5]{imagenes/3-vacios.jpg}
	\end{center}
	\caption{El Señor de los Caballos - Tableros vacíos}\label{fig:3-vacios}
\end{figure}

Observemos que en el gráfico de la Figura \ref{fig:3-vacios} el crecimiento no es lineal y que n=2 es más de 1800 veces más rápido que n=5. De hecho, observemos que en el gráfico, el eje 'y' está en escala logarítmica, y aún así, si unimos los puntos con una línea, da algo mayor que lineal. Esto es un indicio de que nuestro algoritmo es exponencial, lo cual apoya nuestro analisis de complejidad.

\vspace*{0.6cm}

%\newpage
\subsubsection{Test 2}

\vspace*{0.3cm}

En este caso, repetimos la experimentación anterior, pero con los tableros llenos de caballos preubicados (Figura \ref{fig:3-llenos}).

\begin{figure}[htb]
	\begin{center}
    		\includegraphics[scale=0.5]{imagenes/3-llenos.jpg}
	\end{center}
	\caption{El Señor de los Caballos - Tableros vacíos}\label{fig:3-llenos}
\end{figure}

Veamos que la diferencia de tiempos es realmente mínima, y el tiempo de ejecución es realmente pequeño, con un pico en 5. Esto probalemente se deba a que, por como está hecho nuestro algoritmo, si el tablero está lleno, nos ahorramos el costo de recorrerlo y nos limitamos a devolver el tablero de entrada, ya que este es solución. Las diferencias de tiempo (sobre todo la vista en n=5), puede explicarse por el costo de copiar el tablero inicial como tablero solución, que es $\mathcal{O}(n^2)$.

\vspace*{0.6cm}

%\newpage
\subsubsection{Test 3}

\vspace*{0.3cm}

Para este caso, decidimos crear 9 instancias de tableros de 5x5.  Para cada uno, fueron insertándose caballos preubicados de manera aleatoria.  El tablero con 2 caballos preubicados, por ejemplo, tiene el mismo caballo preubicado que el tablero con 1 caballo, y uno más.  Los 9 tableros contemplan desde 0 hasta 8 caballos preubicados.

Nuevamente, para cada instancia se ejecutó el programa 50 veces y se consideró el promedio de los tiempos incurridos (Figura \ref{fig:3-random}).

\begin{figure}[htb]
	\begin{center}
    		\includegraphics[scale=0.5]{imagenes/3-random.jpg}
	\end{center}
	\caption{El Señor de los Caballos - Tableros vacíos}\label{fig:3-random}
\end{figure}

Observamos que hay una abrupta disminución de tiempo por cada caballo agregado. Esto se podría deber a que cuando hay un caballo preubicado, nuestro algoritmo no debe decidir entre poner un caballo o no ponerlo, y se ahorra toda la rama de decisión de poner un caballo. Esto último ya fue abordado en el apartado de análisis de complejidad.

En conclusión, si bien no podemos decir que nuestro algoritmo es exponencial de manera empírica, las dificultades generadas por los tiempos de experimentación y los experimentos, apoyan la teoría de que nuestro algoritmo pueda ser del orden exponencial.


\newpage
\section{Apéndice 1: acerca de los tests}


\newpage
\section{Apéndice 2: secciones relevantes del código}
En esta sección, adjuntamos parte del código correspondiente a la resolución de cada problema que consideramos más \textbf{relevante}.

\subsection{Código del Problema 1}


%\newpage
\subsection{Código del Problema 2}

%\newpage
\subsection{Código del Problema 3}

\end{document}

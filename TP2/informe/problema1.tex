\subsection{Descripción del problema.}

\vspace*{0.3cm}

Tenemos la intención de competir en una versión particular del Rally Dakkar, y para esto, decidimos usar nuestros conocimientos en computación a nuestro a favor. Sabemos que en esta competencia, podremos usar una BMX, una motocross y un buggy arenero. Nuestro objetivo es finalizar la competencia en el menor tiempo posible para nuestras habilidades, y contamos con los siguientes datos:

\begin{itemize}

	\item El circuito se divide en $n$ etapas (numeradas del 1 al n) y en cada etapa solo se puede usar un vehículo
	\item Para cada etapa, se sabe cuanto tardaría en recorrerla con cada vehículo
	\item Tanto la motocross como el buggy arenero tiene un número limitado de veces que se pueden usar, a diferencia de la BMX, y estos números son conocidos
	\item La complejidad del algoritmo pedido es de $\mathcal{O}(n \cdot k_{m} \cdot k_{n})$ donde $n$ es la cantidad de etapas, $k_{m}$ es la cantidad máxima de veces que puedo usar la moto, y $k_{b}$ la cantidad máxima de veces que puedo usar la moto.

Ejemplo:
Supongamos un Rally de 3 etapas, en donde puedo usar una vez la moto, y una vez el buggy, con los siguientes datos:
\begin{itemize}
	\item Si realizo la etapa 1 con la BMX, tardo 10, con la moto 8 y con el buggy 7
	\item Si realizo la etapa 2 con la BMX, tardo 8, con la moto 3 y con el buggy 7
	\item Si realizo la etapa 3 con la BMX, tardo 15, con la moto 6 y con el buggy 2

\end{itemize}

Con estos datos, el menor tiempo en que puedo realizar este Rally es en 15, y viene de tomar la bici en la primer etapa, la moto en la segunda, y el buggy en la tercera.

\end{itemize}

\vspace*{0.6cm}
%\newpage
\subsection{Desarrollo de la idea.}

\vspace*{0.3cm}
Primero, definamos $n$ la cantidad de etapas, $k_{m}$ y $k_{b}$ la máxima cantidad de motos y buggys respectivamente, que puedo utilizar en la competencia (llamo a este estado, ``Estado A"). Supongamos que quiero ver que vehiculo utilizar en la etapa $n$ de manera tal de optimizar mi tiempo. En principio, podríamos usar cualquiera de los tres vehículos: 


\begin{enumerate}
	\item Observemos que en caso de tener que usar la bicicleta, si de alguna manera pudieramos saber que vehiculos utilizar en las anteriores $n-1$ etapas, usando como mucho $k_{m}$ motos y $k_{b}$ buggys, y el tiempo total implementado en este último caso, y a ese tiempo se lo sumaramos al tiempo de usar la bici en la etapa $n$, entonces mi tiempo sería óptimo.
	\item Si la mejor elección fuera la moto, entonces si de alguna manera pudieramos saber que vehiculos utilizar en las anteriores $n-1$ etapas, usando como mucho $k_{m}-1$ motos y $k_{b}$ buggys, y el tiempo total implementado en este último caso, y a ese tiempo se lo sumaramos al tiempo de usar la moto en la etapa $n$, entonces mi tiempo sería el óptimo.(Esto solo es posible, si $k_{m}>0$, puesto que si no, no es posible tomar una moto)
	\item Y si la mejor elección fuera el buggy, entonces si de alguna manera pudieramos saber que vehiculos utilizar en las anteriores $n-1$ etapas, usando como mucho $k_{m}$ motos y $k_{b}-1$ buggys, y el tiempo total implementado en este último caso, y a ese tiempo se lo sumaramos al tiempo de usar el buggy en la etapa $n$, entonces mi tiempo sería el óptimo.(Esto solo es posible, si $k_{b}>0$, puesto que si no, no es posible tomar un buggy)
	 
\end{enumerate}

Entonces, ver cual vehiculo usar en la etapa $n$, es simplemente ver cual es el menor de los tres casos anteriormente expuestos, y elegirlo.Esta decisión será el paso recursivo. 

Pero teníamos tres elementos que dijimos ``si pudieramos saber". Observemoslos más de cerca.

\begin{enumerate}
	\item ``si de alguna manera pudieramos saber que vehiculos utilizar en las anteriores $n-1$ etapas, usando como mucho $k_{m}$ motos y $k_{b}$ buggys, y el tiempo total implementado en este último caso".Pero si tomamos a $n-1$ como si fuera $n$, y dejamos a $k_{m}$ y a $k_{b}$ como están, por recursión podríamos hayar ese vehículo y su tiempo óptimo.
	\item ``si de alguna manera pudieramos saber que vehiculos utilizar en las anteriores $n-1$ etapas, usando como mucho $k_{m}-1$ motos y $k_{b}$ buggys, y el tiempo total implementado en este último caso".Pero si tomamos a $n-1$ como si fuera $n$, y a $k_{m-1}$ como si fuera $k_{m}$ y dejamos a $k_{b}$ como está, por recursión podríamos hallar ese vehículo y su tiempo óptimo.
	\item ``si de alguna manera pudieramos saber que vehiculos utilizar en las anteriores $n-1$ etapas, usando como mucho $k_{m}$ motos y $k_{b}-1$ buggys, y el tiempo total implementado en este último caso".Pero si tomamos a $n-1$ como si fuera $n$, y a $k_{b-1}$ como si fuera $k_{b}$ y dejamos a $k_{m}$ como está, por recursión podríamos hallar ese vehículo y su tiempo óptimo.
\end{enumerate}

Así, yo podría saber que vehículo usar en cada etapa, y cuanto es el tiempo óptimo, con excepción de la primera, puesto que no puede visitar las $n-1$ etapa anteriores. Entonces podemos tomar esto como mi ``caso base".Entonces supongamos que yo tengo una sola etapa:
  \begin{enumerate}
  \item si $k_{m}$ y $k_{b}$ fueran $0$, entonces sencillamente tomo la bicicleta de esa etapa, puesto que no tengo otra opción.
  \item si $k_{b}$ fuera $0$ y $k_{m}$ fuera algún numero entre $1$ y el $k_{m}$ del ``Estado A" inclusive, entonces elijo entre la moto y la bici, la que tarde menos tiempo, y ese es el tiempo de completar la primera etapa.(Debido a que, para una etapa, poder usar una moto, o 500 no es diferencia)
  \item si $k_{m}$ fuera $0$ y $k_{b}$ fuera algún numero entre $1$ y el $k_{b}$ del ``Estado A" inclusive, entonces elijo entre el buggy y la bici, la que tarde menos tiempo, y ese es el tiempo de completar la primera etapa.(Debido a que, para una etapa, poder usar un buggy, o 500 no es diferencia)
  \item si $k_{m}$ fuera algún numero entre $1$ y el $k_{m}$ del ``Estado A" inclusive y $k_{b}$ fuera algún numero entre $1$ y el $k_{b}$ del ``Estado A" inclusive, entonces elijo entre el buggy, la moto y la bici, aquel vehiculo que tenga un tiempo menor, y ese es el tiempo de completar la primera etapa.(Debido a que, para una etapa, poder usar una moto, y un buggy o más de cualquiera de ellos o de ambos, no es diferencia)
  \end{enumerate}
  
   
  
  
  =)
  
   fgjfjtfkrdhrsjrf
   
   hola ale

\vspace*{0.6cm}

%\newpage
\subsection{Demostración de correctitud.}

\vspace*{0.3cm}



\vspace*{0.6cm}

%\newpage
\subsection{Análisis de complejidad.}

\vspace*{0.3cm}


\vspace*{0.6cm}
%\newpage
\subsection{Experimentación y gráficos.}

\vspace*{0.3cm}

\subsubsection{Test 1}

\vspace*{0.3cm}

\vspace*{0.6cm}
%\newpage
\subsubsection{Test 2}

\vspace*{0.3cm}


\subsection{Descripción del problema.}

\vspace*{0.3cm}

Tenemos la intención de competir en una versión particular del Rally Dakkar, y para esto, decidimos usar nuestros conocimientos en computación a nuestro a favor. Sabemos que en esta competencia, podremos usar una BMX, una motocross y un buggy arenero. Nuestro objetivo es finalizar la competencia en el menor tiempo posible para nuestras habilidades, y contamos con los siguientes datos:

\begin{itemize}

	\item El circuito se divide en $n$ etapas (numeradas del 1 al n) y en cada etapa solo se puede usar un vehículo
	\item Para cada etapa, se sabe cuanto tardaría en recorrerla con cada vehículo
	\item Tanto la motocross como el buggy arenero tiene un número limitado de veces que se pueden usar, a diferencia de la BMX, y estos números son conocidos
	\item La complejidad del algoritmo pedido es de $\mathcal{O}(n \cdot k_{m} \cdot k_{n})$ donde $n$ es la cantidad de etapas, $k_{m}$ es la cantidad máxima de veces que puedo usar la moto, y $k_{b}$ la cantidad máxima de veces que puedo usar la moto.
\end{itemize}

Ejemplo:
Supongamos un Rally de 3 etapas, en donde puedo usar una vez la moto, y una vez el buggy, con los siguientes datos:
\begin{itemize}
	\item Si realizo la etapa 1 con la BMX, tardo 10, con la moto 8 y con el buggy 7
	\item Si realizo la etapa 2 con la BMX, tardo 8, con la moto 3 y con el buggy 7
	\item Si realizo la etapa 3 con la BMX, tardo 15, con la moto 6 y con el buggy 2

\end{itemize}

Con estos datos, el menor tiempo en que puedo realizar este Rally es en 15, y viene de tomar la bici en la primer etapa, la moto en la segunda, y el buggy en la tercera.

\vspace*{0.6cm}
%\newpage
\subsection{Desarrollo de la idea y correctitud.}

\vspace*{0.3cm}

LLamemos $n$ a la cantidad de etapas del Rally, $k_{m}$ y $k_{b}$ a la cantidad máxima de motos y buggys respectivamente que es posible utilizar en el Rally. La idea es en vez de tratar de resolver todas las etapas de un tirón, primero tratemos de resolver la primer etapa, luego resolvamos las dos primeras etapas ayudandonos con la resolución de la primera etapa sola, luego resolvamos las primeras tres etapas ayudandonos con la resolución de las primeras dos etapas, y así hasta llegar a $n$. Para esto, la resolución de la primera etapa tendrá una tabla, la resolución de las primeras dos etapas tendrá otra tabla que usará información de la tabla de la primer etapa, y de igual manera para el resto. Cada una de estas tablas tiene $k_{m}+1$ columnas (numeradas del $0$ al $k_{m}$) y $k_{b}+1$ filas(numeradas del $0$ al $k_{b}$). Para hacer referencia a la tabla que representa a las primeras $t$ etapas(con $t$ algún número natural distinto de 0), la llamaremos la tabla $t$.Entonces si tomamos $i$,$j$,$h$ tal que $0\leq i \leq k_{b}$ , $0\leq i \leq k_{m}$ y $1\leq h \leq n$ la idea sería que la posición $i$ $j$ de la tabla de las primeras $h$ etapas, tenga el tiempo óptimo de las primeras $h$ etapas usando a lo sumo $i$ buggys y $j$ motos, y el vehículo que se usaría en la etapa $h$ bajo esas circunstancias(pondremos 1 si es la bici, 2 si es la moto y 3 para buggy). Pero entonces, como se completarían las tablas? Las tablas se completarían de la siguiente manera(para hacer referencia a una posición de una tabla $h$, diremos la posición $h_{a,c}$ donde $a$ es el número de fila, y $c$ el número de columna):

\begin{itemize}
	\item Si $h=1$
	\begin{itemize}
		\item La posición $h_{0,0}$ hay que ponerle el tiempo de la bicicleta en la etapa $h$, puesto que solo puede usarse ese vehículo, y un 1 para indicar que se usó la bici.
		\item La posición $h_{0,1}$ hay que ponerle el tiempo de la bicicleta o la moto(ambos en la etapa $h$), el que sea menor, y depende cual haya elegido, el número que corresponde.
		\item De la posición $h_{0,2}$ hasta $h_{0,k_{m}}$ hay que poner exactamente lo que hay en $h_{0,1}$ dado que si tengo una sola etapa, poder usar una moto o más de una, no es diferencia.
		\item La posición $h_{1,0}$ hay que ponerle el tiempo de la bicicleta o el buggy(ambos de la etapa $h$), el que sea menor y depende cual haya elegido, el número que corresponde.
		\item De la posición $h_{2,0}$ hasta $h_{k_{b},0}$ hay que poner exactamente lo que hay en $h_{1,0}$ dado que si tengo una sola etapa, poder usar un buggy o más de uno, no es diferencia.
		\item La posición $h_{1,1}$ hay que ponerle el tiempo de la bicicleta, el buggy o la moto(de la etapa $h$), el que sea menor y depende cual haya elegido, el número que corresponde.
		\item El resto de las posiciones de esa matriz, deben tener lo que hay en $h_{1,1}$ dado que si tengo una sola etapa, poder usar una moto y un buggy, o más de alguno de ellos o de ambos, no es diferencia.

	\end{itemize}

	\item Para todo $h$ tal que $1 < h \leq n$

	\begin{itemize}

		\item La posición $h_{0,0}$ debe tener el tiempo que hay en la posición $(h-1)_{0,0}$ sumada al de la bici de la etapa $h$, y un 1 para indicar que se usó la bici. El tiempo óptimo de este caso es correcto, puesto que al no poder usar otros vehículos, el tiempo óptimo de haber recorrido $h$ etapas usando solo la bicicleta, es el tiempo óptimo de haber recorrido $h-1$ etapas sin usar buggys ni motos, sumado a usar una bicicleta en la etapa $h$. El vehículo elegido es trivialmente correcto, puesto que no hay otra elección posible. 
		\item Tomando $1\leq j \leq k_{m}$ la posición $h_{0,j}$ debe contener el tiempo mínimo entre:
		\begin{enumerate}
			\item El tiempo de $(h-1)_{0,j}$ sumado al tiempo de la bici en la etapa $h$
			\item El tiempo de $(h-1)_{0,(j-1)}$ sumado al tiempo de la moto en la etapa $h$
		\end{enumerate} 
		Y según cual se haya tomado, el número correspondiente a ese vehículo. El tiempo óptimo es correcto puesto que 1) significa que si el vehículo que deberíamos tomar fuera la bicicleta, entonces en las anteriores $h-1$ etapas puedo usar las $j$ motos, dado que no voy a usar ninguna moto en esta etapa y tomo el tiempo óptimo de eso. Por otro lado, 2) significa que si el vehículo que deberíamos tomar es la moto, entonces en las anteriores $h-1$ etapas, debo usar $j-1$ motos para poder usar una moto ahora, y tomo el tiempo óptimo de eso. Como tomamos el mínimo entre 1) y 2) entonces, y ambos, de ser la respuesta correcta, serían óptimos, entonces mi elección resulta óptima en tiempo. Por esto mismo, el vehículo elegido es el correcto. Ver que no se crean conflictos con el buggy, ni tengo que pensar en si puedo elegirlo o no, puesto que no puedo usar buggys en estos casos.
		\item Tomando $1\leq i \leq k_{b}$ la posición $h_{i,0}$ debe contener el tiempo mínimo entre:
		\begin{enumerate}
			\item El tiempo de $(h-1)_{i,0}$ sumado al tiempo de la bici en la etapa $h$
			\item El tiempo de $(h-1)_{(i-1),0}$ sumado al tiempo del buggy en la etapa $h$
		\end{enumerate} 
		Y según cual se haya tomado, el número correspondiente a ese vehículo.El tiempo óptimo es correcto puesto que 1) significa que si el vehículo que deberíamos tomar fuera la bicicleta, entonces en las anteriores $h-1$ etapas puedo usar las $i$ buggys, dado que no voy a usar ningunn buggy en esta etapa y tomo el tiempo óptimo de eso. Por otro lado, 2) significa que si el vehículo que deberíamos tomar es el buggy, entonces en las anteriores $h-1$ etapas, debo usar $i-1$ buggys para poder usar uno en esta etapa, y tomo el tiempo óptimo de eso. Como tomamos el mínimo entre 1) y 2) entonces, y ambos, de ser la respuesta correcta, serían óptimos, entonces mi elección resulta óptima en tiempo. Por esto mismo, el vehículo elegido es el correcto. Ver que no se crean conflictos con la moto, ni tengo que pensar en si puedo elegirla o no, puesto que no puedo usar motos en estos casos.
		\item Tomando $1\leq i \leq k_{b}$ y $1\leq j \leq k_{m}$ la posición $h_{i,j}$ debe contener el tiempo mínimo entre:
		\begin{enumerate}
			\item El tiempo de $(h-1)_{i,j}$ sumado al tiempo de la bici en la etapa $h$
			\item El tiempo de $(h-1)_{i,(j-1)}$ sumado al tiempo de la moto en la etapa $h$
			\item El tiempo de $(h-1)_{(i-1),j}$ sumado al tiempo del buggy en la etapa $h$
		\end{enumerate} 
		Y según cual se haya tomado, el número correspondiente a ese vehículo. El tiempo óptimo es correcto puesto que 1) significa que si el vehículo que deberíamos tomar fuera la bicicleta, entonces en las anteriores $h-1$ etapas puedo usar las $j$ motos e $i$ buggys, dado que no voy a usar ninguna moto ni ningún buggy en esta etapa y tomo el tiempo óptimo de eso. Por otro lado, 2) significa que si el vehículo que deberíamos tomar es la moto, entonces en las anteriores $h-1$ etapas, debo usar $j-1$ motos para poder usar una moto ahora, y como no utilizo el buggy, puedo usar los $i$ que tengo, y tomo el tiempo óptimo de eso.Por otra parte, 3) significa que si el vehículo que deberíamos tomar es el buggy, entonces en las anteriores $h-1$ etapas, debo usar $i-1$ buggys para poder usar uno en esta etapa,y como no utilizo el buggy, puedo usar los $i$ que tengo,y tomo el tiempo óptimo de eso. Como tomamos el mínimo entre 1) , 2) y 3), y todos, de ser la respuesta correcta, serían óptimos, entonces mi elección resulta óptima en tiempo. Por esto mismo, el vehículo elegido es el correcto.

	\end{itemize}

\end{itemize}
	Entonces, una vez llenas las tablas si quiero saber cual es el menor tiempo en el que puedo completar el Rally, tan solo debo ver el tiempo que hay en $n_{k_{b},k_{m}}$, y si quiero saber que vehículos utilicé en cada etapa, debo ver que vehículo se usó en $n_{k_{b},k_{m}}$ y ese vehículo será el que se usó en la etapa $n$. Luego, de manera general, suponiendo $2 \leq h \leq n$, para extraer el vehículo utilizado en la etapa $h-1$ habiendo extraído el vehículo de la posición h de la posición $h_{i,j}$ con $0\leq i \leq k_{b}$ y $0\leq j \leq k_{m}$, por como está hecha la tabla, debo ir a:
	
\begin{itemize}
	\item la posición $(h-1)_{i,j}$ si en la etapa $h$ elegí la bici.
	\item la posición $(h-1)_{i,j-1}$ si en la etapa $h$ elegí la moto.
	\item la posición $(h-1)_{i-1,j}$ si en la etapa $h$ elegí el buggy.
\end{itemize}			

	 Y de la posición que haya ido, extraer el vehículo. Este será el vehículo utilizado en la etapa $h-1$. Haciendo esto en orden, se debería obtener que vehículo fue utilizado en cada etapa, y con todo esto se resolvería el problema de manera correcta.
 
\vspace*{0.6cm}

%\newpage

\subsection{Análisis de complejidad.}

\vspace*{0.3cm}


\vspace*{0.6cm}
%\newpage
\subsection{Experimentación y gráficos.}

\vspace*{0.3cm}

\subsubsection{Test 1}

\vspace*{0.3cm}

\vspace*{0.6cm}
%\newpage
\subsubsection{Test 2}

\vspace*{0.3cm}


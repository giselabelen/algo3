\subsection{Desarrollo de la idea.}

\vspace*{0.3cm}

Sea $G$ un grafo cualquiera, $I$ un conjunto independiente de ese nodo, y $n_{1}$,$n_{2}$ nodos de G que no pertenecen a $I$ y no tienen aristas en común con ningún elemento de $I$, diremos que $n_{1}$ es óptimo si no existe ningún $n_{2}$ tal que $( \# (Vecinos(n_{2})) - \# (Vecinos(n_{2}) \cap Vecinos(I))) > ( \# (Vecinos(n_{1})) - \# (Vecinos(n_{1}) \cap Vecinos(I)))$. Definimos $Vecinos(\alpha)$ como el conjunto de nodos a los cuales $\alpha$ lleva una arista si $\alpha$ es un nodo, y si $\alpha$ es un conjunto, entonces es el conjunto de nodos a los cuales les llega un arista desde por lo menos un elemento de $\alpha$.  De manera más informal, podemos decir que un nodo óptimo es aquél que más vecinos ``libres'' tiene, siendo un nodo ``libre'' uno que no está en el conjunto solución ni es adyacente a un nodo de la solución.

Nuestra heurística se basa en formar un conjunto independiente maximal de la siguiente manera:
Primero, considerando al conjunto independiente vacío, tomamos a un nodo óptimo, y lo agregamos como nuevo elemento de nuestro conjunto independiente. Luego con nuestro nuevo conjunto independiente, tomamos un nodo óptimo, y lo agregamos a nuestro conjunto independiente. Repetimos esto hasta que no exista un nodo óptimo, puesto que nuestro conjunto independiente se transformó en maximal, y por lo tanto en un conjunto dominante.
 
\vspace*{0.6cm}

%\newpage

\subsection{Análisis de complejidad.}

\vspace*{0.3cm}

\begin{figure}
\begin{codebox}
\Procname{$\proc{CIDM_goloso}(lista\_nodos$ $cidm\_sol)$} 
\li $res \leftarrow$ 0
\li \While no se hayan ``tomado'' todos los nodos
\li 	\Do 
 		$elegido \leftarrow$ nodo ``óptimo''
\li 		$cidm\_sol \leftarrow$ agregar $elegido$
\li 		incrementar $res$
\li 		marcar a $elegido$ y a sus vecinos como ``tomados''
	\End
\li \Return $res$
\end{codebox}
\caption{Heurística golosa constructiva para CIDM}\label{code:goloso}
\end{figure}
%\FloatBarrier

\vspace*{0.6cm}
%\newpage

\subsection{Instancias no óptimas.}

\vspace*{0.3cm}

A continuación presentaremos dos familias de grafos para las cuales nuestra heurística golosa no siempre encuentra una solución óptima.

\subsubsection*{Familia 1 - Estrellas}

La forma de los grafos pertenecientes a esta familia responde a las siguientes características:

\begin{itemize}
\item Existe un único vértice $v$ con grado máximo.  Sea $\delta_{max}$ el grado de este nodo.
\item Para cada vértice $w$ adyacente a $v$, $\delta(w) < \delta_{max}$, y sus nodos adyacentes (salvo $v$) tienen grado 1.	
\end{itemize}

Dado que el algoritmo propuesto va tomando en cada paso el nodo ``óptimo'', o sea, aquél que más nodos ``libres'' cubre, en un grafo como el descrito primero tomará al vértice $v$ con grado máximo $\delta_{max}$.  Luego, para cumplir con la independencia y la dominancia, deberá tomar a los vértices adyacentes a los vecinos de $v$. Como el grado de cada vecino de $v$ es menor a $\delta_{max}$, el peor de los casos sería que cada uno de ellos tuviera grado $\delta_{max} - 1$.  En este caso, cada vértice adyacente a $v$ tendría $\delta_{max} - 2$ vecinos además del mismo $v$, y entonces la solución hallada por nuestra heurística estaría conformada por $\delta_{max} \times (\delta_{max} - 2)$ nodos.  Sin embargo, para un grafo como el detallado, existe una solución mejor conformada por $\delta_{max}$ nodos, y corresponde al conjunto compuesto por los vecinos del nodo $v$.

La Figura \ref{fig:familia1} muestra un ejemplo de grafo perteneciente a esta familia, y la Figura \ref{fig:familia1res} muestra en color verde la solución hallada por nuestro algoritmo, la cual claramente es peor que la solución formada por los vértices que quedaron en rojo en la misma Figura.

\begin{figure}[!htb]
\minipage{0.5\textwidth}
\begin{center}
  \includegraphics[scale=0.5]{imagenes/familia1.png}
\end{center}
  \caption{Grafo perteneciente a la Familia 1}\label{fig:familia1}
\endminipage\hfill
\minipage{0.5\textwidth}
\begin{center}
  \includegraphics[scale=0.5]{imagenes/familia1-res.png}
\end{center}
  \caption{Solución hallada por nuestra heurística}\label{fig:familia1res}
\endminipage
\end{figure}

\subsubsection*{Familia 2 - Circuitos}

Como miembros de esta familia consideraremos a los circuitos simples.  Para este tipo de grafos, la calidad de la solución hallada por nuestro algoritmo dependerá de la forma en que se hayan rotulado los vértices.

Tomemos el grafo $C_{n}$ formado por los vértices $v_1,v_2,...,v_n$.

\begin{itemize}
\item Si los vértices se relacionan de manera que $v_1$ se relaciona con $v_n$ y $v_2$, $v_n$ se relaciona con $v_{n-1}$ y $v_1$, y para todo $1 < i < n$ el vértice $v_i$ se relaciona con los vértices $v_{i-1}$ y $v_{i+1}$, y si consideramos que nuestro algoritmo recibe los nodos $v_i$ en orden creciente de $i$, entonces formará la solución de la siguiente manera:

	\begin{itemize}
	\item Tomará a $v_1$.
	\item Luego, $v_2$ no podrá ser tomado por ser adyacente a $v_1$.
	\item Si $n = 3$, $v_3$ también quedaría cubierto por $v_1$ y el conjunto solución sería $\{v_1\}$.  
	\item Si $n = 4$, $v_4$ estaría cubierto por $v_1$ y el único vértice ``libre'' restante sería $v_3$, por lo cual sería tomado y el conjunto solución sería $\{v_1,v_3\}$.
	\item Si $n > 4$, $v_3$ no sería un nodo óptimo, ya que uno de sus nodos adyacentes ($v_2$) ya fue cubierto por $v_1$, mientras que $v_4$ sí sería óptimo dado que podría cubrir a $v_3$ y a $v_5$, así que $v_4$ pasaría a formar parte de la solución.
	\item Los demás nodos del conjunto solución irán siendo tomados siguiendo los mismos pasos.
	\end{itemize}
	
	Podríamos decir, entonces, que la solución se genera tomando los vértices $v_j$ con $j = 1 + 3 \times k$ ($k$ entero no negativo) siempre que $j < n-1$. [VER COMO ENCHUFAR LO DE UN TERCIO]
	
\item Si los vértices se relacionan... [NO SE COMO ESCRIBIR EL CASO FEO]
\end{itemize}

\begin{figure}[!htb]
\minipage{0.5\textwidth}
\begin{center}
  \includegraphics[scale=0.8]{imagenes/faimilia2.png}
\end{center}
  \caption{Circuito simple rotulado ``en orden''}\label{fig:familia2}
\endminipage\hfill
\minipage{0.5\textwidth}
\begin{center}
  \includegraphics[scale=0.8]{imagenes/faimilia2-resopt.png}
\end{center}
  \caption{Solución hallada por nuestra heurística}\label{fig:familia2res}
\endminipage\\
\minipage{0.5\textwidth}
\begin{center}
  \includegraphics[scale=0.8]{imagenes/faimilia2rename.png}
\end{center}
  \caption{Circuito simple rotulado ``no en orden''}\label{fig:familia2bis}
\endminipage\hfill
\minipage{0.5\textwidth}
\begin{center}
  \includegraphics[scale=0.8]{imagenes/faimilia2rename-solnoopt.png}
\end{center}
  \caption{Solución hallada por nuestra heurística}\label{fig:familia2bisres}
\endminipage
\end{figure}



\vspace*{0.6cm}

\subsection{Experimentación y gráficos.}

\vspace*{0.3cm}

\vspace*{0.6cm}

\subsubsection{Test 1}
\vspace*{0.3cm}

\vspace*{0.6cm}
%\newpage

\subsubsection{Test 2}


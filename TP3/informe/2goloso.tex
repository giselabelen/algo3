\subsection{Desarrollo de la idea.}

\vspace*{0.3cm}

Sea $G$ un grafo cualquiera, $I$ un conjunto independiente de ese nodo, y $n_{1}$,$n_{2}$ nodos de G que no pertenecen a $I$ y no tienen aristas en común con ningún elemento de $I$, diremos que $n_{1}$ es óptimo si no existe ningún $n_{2}$ tal que $( \# (Vecinos(n_{2})) - \# (Vecinos(n_{2}) \cap Vecinos(I))) > ( \# (Vecinos(n_{1})) - \# (Vecinos(n_{1}) \cap Vecinos(I)))$. Definimos $Vecinos(\alpha)$ como el conjunto de nodos a los cuales $\alpha$ lleva una arista si $\alpha$ es un nodo, y si $\alpha$ es un conjunto, entonces es el conjunto de nodos a los cuales les llega un arista desde por lo menos un elemento de $\alpha$.

Nuestra heurística se basa en formar un conjunto independiente maximal de la siguiente manera:
Primero, considerando al conjunto independiente vacío, tomamos a un nodo óptimo, y lo agregamos como nuevo elemento de nuestro conjunto independiente. Luego con nuestro nuevo conjunto independiente, tomamos un nodo óptimo, y lo agregamos a nuestro conjunto independiente. Repetimos esto hasta que no exista un nodo óptimo, puesto que nuestro conjunto independiente se transformó en maximal, y por lo tanto en un conjunto dominante.
 
\vspace*{0.6cm}

%\newpage

\subsection{Análisis de complejidad.}

\vspace*{0.3cm}

\begin{figure}
\begin{codebox}
\Procname{$\proc{CIDM_goloso}(lista\_nodos$ $cidm\_sol)$} 
\li $res \leftarrow$ 0
\li \While no se hayan ``tomado'' todos los nodos
\li 	\Do 
 		$elegido \leftarrow$ nodo ``óptimo''
\li 		$cidm\_sol \leftarrow$ agregar $elegido$
\li 		incrementar $res$
\li 		marcar a $elegido$ y a sus vecinos como ``tomados''
	\End
\li \Return $res$
\end{codebox}
\caption{Heurística golosa constructiva para CIDM}\label{code:goloso}
\end{figure}
%\FloatBarrier

\vspace*{0.6cm}
%\newpage

\subsection{Instancias no óptimas.}


\subsection{Experimentación y gráficos.}

\vspace*{0.3cm}


\subsubsection{Test 1}
\vspace*{0.3cm}

\vspace*{0.6cm}
%\newpage

\subsubsection{Test 2}


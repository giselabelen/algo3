\subsection{Desarrollo de la idea.}

\vspace*{0.3cm}

GRASP (Greedy Randomized Adaptative Search Procedure) es una combinación entre una heurística golosa ``aleatorizada'' y un procedimiento de búsqueda local.  La idea es la siguiente:

\begin{codebox}
\li \While no se alcance el {\it criterio de terminación}
\li \Do 
		Obtener una solución inicial mediante una {\it heurística golosa aleatorizada}.
\li 		Mejorar la solución mediante búsqueda local.
\li 		Recordar la mejor solución obtenida hasta el momento.
	\End
\end{codebox}

Como criterios de terminación, analizaremos dos casos particulares:

\begin{itemize}
\item Se realizaron $k$ iteraciones.
\item Se realizaron $k$ iteraciones sin encontrar una solución mejor.
\end{itemize}

En cuanto a la heurística golosa aleatorizada, hemos continuado con la idea del algoritmo goloso descrito anteriormente, con la variación de que, en cada paso, se genera una Lista Restricta de Candidatos (RCL) y se elige aleatoriamente un candidato de esa lista.  Analizaremos dos posibles maneras de construir dicha RCL:

\begin{itemize}
\item Hallar al mejor candidato, es decir, el nodo que hemos definido como óptimo, y colocar en la RCL los nodos candidatos cuya cantidad de vecinos ``libres'' sea no menor a un cierto porcentaje $\alpha$ de la cantidad de vecinos ``libres'' del mejor candidato.
\item Hallar al mejor candidato y colocar en la RCL los $\beta$ mejores candidatos, es decir, los $\beta$ nodos que más nodos ``libres'' cubran.
\end{itemize}

\vspace*{0.6cm}

%\newpage
\subsection{Experimentación y gráficos.}

\vspace*{0.3cm}


\subsubsection{Test 1}
\vspace*{0.3cm}

\vspace*{0.6cm}
%\newpage

\subsubsection{Test 2}


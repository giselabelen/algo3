\subsection{Desarrollo de la idea.}

\vspace*{0.3cm}

Una mejora propuesta al algoritmo goloso explicado anteriormente, es llevar a cabo un procedimiento de búsqueda local.

A partir de una solución hallada por dicha heurística golosa, se irán encontrando nuevas soluciones ``vecinas''.  Si una solución ``vecina'' resulta ser mejor, es decir, logra encontrar un conjunto independiente maximal con menos elementos que la solución anterior, entonces la reemplazará.  Estos pasos se repitirán hasta que se encuentre una solución que no pueda mejorarse, es decir, una solución óptima local.

Plantearemos dos ``vecindades'' posibles para las soluciones:

\begin{itemize}
\item {\bf Vecindad 1:}
\item {\bf Vecindad 2:}
\end{itemize}

 
\vspace*{0.6cm}

%\newpage

\subsection{Análisis de complejidad de una iteración.}

\vspace*{0.3cm}

\begin{figure}
\begin{codebox}
\Procname{$\proc{CIDM_busqueda}(int$ $mej)$}
\li $cidm\_sol \leftarrow$ lista vacía de nodos
\li $res \leftarrow$ {\sc CIDM_goloso}($cidm\_sol$)
\li \While haya mejoras y la última solución tenga más de un nodo
\li \Do 
		\If $mej == 1$
\li 		\Then {\sc mejorador1}($cidm\_sol,res$)
\li 		\Else {\sc mejorador2}($cidm\_sol,res$)
		\End
	\End
\end{codebox}
\caption{Heurística de búsqueda local para CIDM}\label{code:busqueda}
\end{figure}
%\FloatBarrier


\begin{figure}
\begin{codebox}
\Procname{$\proc{Mejorador1}(lista\_nodos$ $cidm\_sol,int$ $res)$} 
\li \For cada par de nodos en $cidm\_sol$
\li \Do 
		``sacar'' ambos nodos
\li 		\For cada vecino $n$ de estos nodos
\li 		\Do 
			\If $n$ quedó ``desconectado''
\li			\Then
				``agregar'' $n$
\li 				\If se forma una solución válida
\li 				\Then salir del ciclo
				\End
			\End
\li 			\If se encontró una solución mejor
\li 			\Then
				actualizar $cidm\_sol$
\li 				actualizar $res$
\li 				\Return
			\End
		\End
	\End
\li \Return
\end{codebox}
\caption{Pseudocódigo de la mejora 1}\label{code:mej1}
\end{figure}
%\FloatBarrier



\begin{figure}
\begin{codebox}
\Procname{$\proc{Mejorador2}(lista\_nodos$ $cidm\_sol,int$ $res)$} 
\li \For cada nodo $n$
\li \Do 
		\If $n$ se conecta con al menos dos nodos de $cidm\_sol$
\li 		\Then 
			\For cada nodo de $cidm\_sol$ que se conectan $n$
\li 			\Do 
				``sacar'' el nodo
\li 				\If se forma una solución válida
\li 				\Then salir del ciclo
				\End
			\End
\li 			\If se encontró una solución mejor
\li 			\Then
				actualizar $cidm\_sol$
\li 				actualizar $res$
\li 				\Return
			\End
		\End
	\End
\li \Return
\end{codebox}
\caption{Pseudocódigo de la mejora 2}\label{code:mej2}
\end{figure}
%\FloatBarrier


\vspace*{0.6cm}
%\newpage
\subsection{Experimentación y gráficos.}

\vspace*{0.3cm}


\subsubsection{Test 1}
\vspace*{0.3cm}

\vspace*{0.6cm}
%\newpage

\subsubsection{Test 2}


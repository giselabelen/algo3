\documentclass[a4paper]{article}
\usepackage[spanish]{babel}
\usepackage[utf8]{inputenc}
\usepackage{fancyhdr}
\usepackage{charter}   % tipografía
\usepackage{graphicx}
\usepackage{makeidx}

\usepackage{float}
\usepackage{amsmath, amsthm, amssymb}
\usepackage{amsfonts}
\usepackage{sectsty}
\usepackage{wrapfig}
\usepackage{listings} % necesario para el resaltado de sintaxis
\usepackage{caption}
\usepackage{placeins}

\usepackage{hyperref} % agrega hipervínculos en cada entrada del índice
\hypersetup{          % (en el pdf)
    colorlinks=true,
    linktoc=all,
    citecolor=black,
    filecolor=black,
    linkcolor=black,
    urlcolor=black
}

\input{codesnippet}
\input{page.layout}
\usepackage{underscore}
\usepackage{caratula}
\usepackage{url}
\usepackage{color}
\usepackage{clrscode3e} % necesario para el pseudocodigo (estilo Cormen)




\begin{document}

\lstset{
  language=C++,                    % (cambiar al lenguaje correspondiente)
  backgroundcolor=\color{white},   % choose the background color
  basicstyle=\footnotesize,        % size of fonts used for the code
  breaklines=true,                 % automatic line breaking only at whitespace
  captionpos=b,                    % sets the caption-position to bottom
  commentstyle=\color{red},    % comment style
  escapeinside={\%*}{*)},          % if you want to add LaTeX within your code
  keywordstyle=\color{blue},       % keyword style
  stringstyle=\color{blue},     % string literal style
}

\thispagestyle{empty}
\materia{Algoritmos y Estructuras de Datos III}
\submateria{Primer Cuatrimestre de 2015}
\titulo{Heurísticas y Metaheurísticas}
\subtitulo{CIDM}
\integrante{Barañao, Facundo}{480/11}{facundo_732@hotmail.com}
\integrante{Confalonieri, Gisela Belén}{511/11}{gise_5291@yahoo.com.ar} % por cada integrante (apellido, nombre) (n° libreta) (e-mail)
\integrante{Mignanelli, Alejandro Rubén}{609/11}{minga_titere@hotmail.com} 

\maketitle
\newpage

\thispagestyle{empty}
\vfill
%\begin{abstract}
%    \vspace{0.5cm}
%	
%
%\end{abstract}

\thispagestyle{empty}
\vspace{1.5cm}
\tableofcontents
\newpage

%\normalsize
 
\newpage

\section{Introducción}

\vspace*{0.3cm}

En el presente trabajo, se pretende analizar y comparar diferentes maneras de encarar el problema de {\it Conjunto Independiente Dominante Mínimo (CIDM)}, el cual consiste en hallar un conjunto independiente dominante de un grafo, con mínima cardinalidad. En particular se utilizará un algoritmo exacto, una heurística golosa, una heurística de busqueda local, y la metahuristica de GRASP.

A continuación se presentan algunas definiciones que nos serán útiles para comprender y abordar el problema:

\begin{itemize}

\item Sea $G = (V, E)$ un grafo simple. Un conjunto $D \subseteq V$ es un conjunto dominante de $G$ si todo vértice de $G$ está en $D$ o bien tiene al menos un vecino que está en $D$. Por otro lado, un conjunto $I \subseteq V$ es un conjunto independiente de $G$ si no existe ningún eje de $E$ entre dos vértices de $I$. Definimos entonces un conjunto independiente dominante de $G$ como un conjunto independiente que a su vez es un conjunto dominante del grafo $G$.

\item Un conjunto independiente de $I \subseteq V$ se dice maximal si no existe otro conjunto independiente $J \subseteq V$ tal que $I \subset J$, es decir tal que $I$ está incluido estrictamente en $J$. %En otra sección de este informe se demostrará que todo conjunto independiente dominante, es particularmente independiente maximal.

\end{itemize}

\vspace*{0.6cm}

\section{Relación con trabajos previos y aplicaciones}

\vspace*{0.3cm}

En el TP1 de la materia, hemos encontrado y analizado un algoritmo que resolvía un problema que fue llamado {\bf El Señor de los Caballos}. El problema era el siguiente:

\vspace*{0.1cm}

{\it Se tiene un juego de mesa cuyo tablero, dividido en casillas, posee igual cantidad de filas y columnas y hace uso de una conocida pieza del popular ajedrez: el caballo. El juego es solamente para un jugador y consiste en, teniendo caballos ubicados en distintos casilleros, insertar en casilleros vacíos la mínima cantidad de caballos extras, de manera tal que, siguiendo las reglas del movimiento de los caballos en el ajedrez, todas las casillas se encuentren ocupadas o amenazadas por un caballo.
Aspectos a tener en cuenta:

\begin{itemize}
   \item Se conoce la cantidad de filas y columnas del tablero.
   \item Se conoce la cantidad de caballos que ocupan el tablero inicialmente.
   \item Para cada uno de estos caballos, se sabe su ubicación en el tablero.
   \item Una casilla se considera amenazada si existe un caballo tal que en una movida pueda ocupar dicha casilla.
\end{itemize}
}

\vspace*{0.1cm}

Observemos que si consideramos a cada casilla como un nodo, y que dos nodos son adyacentes cuando un caballo puede llegar de uno a otro con un movimiento, entonces {\bf El Señor de los Caballos} puede ser visto como la búsqueda de un conjunto dominante mínimo que contenga a los nodos/casillas que tienen un caballo preubicado. Sin embargo, dado un grafo, no es correcto decir que la cardinalidad del CIDM es menor o igual a la cardinalidad de un conjunto. Para probar esto, podemos observar la figura X. Corriendo el backtracking de CIDM que hemos creado (y que detallaremos en próximas secciones) la solución es la figura X2, mientras que la Figura X3 es el conjunto dominante mínimo. Entonces, no podemos afirmar que El Señor de los Caballos pueda adaptarse a un problema de CIDM. 

\vspace*{0.1cm}

De todos modos, cabe preguntarse en qué situaciones de la vida real podría aplicarse este problema. Veamos algunos ejemplos realistas y alguno no tanto.


\paragraph{Gaseosas:} Una empresa de bebidas gaseosas tiene sus actividades divididas en áreas, las cuales se interrelacionan de cierta manera (no necesariamente todas con todas).  Esta empresa está por sacar una nueva bebida sabor cola y quiere hacerlo antes que su rival, quien está más adelantado en la producción.  Para lograr sacar el producto antes que su contrincante, nuestra empresa debe lograr una mejora en el trabajo de sus diferentes áreas.  Por cómo está estructurada la empresa, si un área tiene una mejora considerable en tiempo, todas las áreas que se relacionan con ella se verán beneficiadas, pero este beneficio ya no impacta en las áreas relacionadas con estas últimas.  Queremos entonces lograr impulsar mejoras considerables en las áreas que sean necesarias para que toda la empresa funcione más rápido y logre sacar su producto a tiempo.  Para ello se contratarán especialistas con sueldos sumamente importantes, por lo cual se desea que la cantidad de áreas a mejorar considerablemente sea mínima (para minimizar la inversión en especialistas).  Además, sabemos que estos especialistas son excelentes, pero muy soberbios/tercos, y ponerlos a cargo de áreas relacionadas puede devenir en discusiones que retrasarían a la empresa, por lo cual procuraremos colocarlos en áreas "no adyacentes".

\paragraph{Detectives:} Se tiene un caso de asesinato, y debemos resolverlo. Nuestras investigaciones previas nos han dado información de todas las personas que tuvieron algo que ver con el incidente, y sabemos cuáles se conocen entre sí y cuáles no, además de que sabemos que toda persona que conoce a otra, sabe lo que hizo dicha persona el día del incidente. Debido a que somos detectives principiantes, el alquiler por día de nuestra oficina nos sale caro, y entrevistar a un testigo, independientemente del volumen de información obtenida, nos toma un día. Si para resolver el caso debiéramos escuchar información sobre todas las personas involucradas, ¿a quienes deberíamos llamar de testigos, de manera tal que reduzcamos el alquiler de oficina al mínimo? (Estos testigos no pueden conocerse entre sí, para evitar complot y falsos datos).

\paragraph{Caballeros del Zodíaco:} Milo de Escorpio\footnote{\url{http://es.wikipedia.org/wiki/Milo_de_Escorpio}}, guardián de la casa de Escorpio, ha recibido muchas quejas de parte del gran patriarca, debido a que por su casa pasa todo el mundo. Cansado de ser tantas veces derrotado, nos pide ayuda, y para esto nos explica la verdad sobre su gran técnica, la aguja escarlata. Lejos de lo que se cuenta en el animé {\it Los caballeros del zodíaco}\footnote{\url{http://es.wikipedia.org/wiki/Saint_Seiya}}, el verdadero poder de la aguja escarlata es el siguiente:

Milo nos explica que el cuerpo de cada ser humano puede ser dividido en sectores energéticos.  Estos sectores energéticos pueden tener una correspondencia entre sí, las cuales no son necesariamente a nivel local (por ejemplo, una porcion del dedo indice de un humano, puede estar conectada a una porción de la cabeza). %Por lo tanto, estos sectores energéticos pueden ser fácilmente vistos como un grafo. 
Cuando la aguja escarlata toca un sector energético de un cuerpo, éste y todos aquéllos sectores que tengan una correspondencia, se inmovilizan. Pero esto sucede sólo si la aguja escarlata impacta contra un sector energético ``sano''. Si el sector en cuestión ya estuviese inmovilizado, entonces la aplicación de la aguja escarlata no hace ningun efecto.

Dado que el uso de una aguja escarlata, resulta en un fuerte uso del cosmos de Milo, se nos pide que, dado un oponente y la información sobre todos sus sectores energéticos y correspondencias, nosotros le fabriquemos un algoritmo tal que le diga en qué sectores del cuerpo debe usar la aguja escarlata, de manera tal que deba utilizar la mínima cantidad de agujas escarlata posibles. Milo nos asegura que el tiempo de dicho algoritmo no tiene importancia, dado que le pidió prestada la habitación del tiempo\footnote{\url{http://es.dragonball.wikia.com/wiki/Habitacion_del_Tiempo}} a Kamisama\footnote{\url{http://es.wikipedia.org/wiki/Kamisama_(personaje)}}, que está equipada con una notebook y un enchufe para dejar la bateria cargada (eso sí, no tiene wifi dado a una muy mala señal, por lo que debemos darle un pendrive con el código).

%\newpage

\vspace*{0.6cm}

\section{Propiedades}

\vspace*{0.3cm}

En esta sección demostraremos ciertas propiedades que serán usadas para elaborar una conclusión que nos ayudará a abordar el problema propuesto.

\subsection{Todo conjunto independiente maximal es dominante.}
Lo probaremos por absurdo.  Supongamos que existe algún grafo $G = (V,E)$ tal que tiene un conjunto independiente maximal $C$ que no es un conjunto dominante. Como $C$ no es dominante, entonces existe un nodo $v \in V$ tal que $v \not \in C$, y que además dentro del grafo original, $v$ no es vecino de ningún elemento de $C$. Pero entonces, si añadimos a $v$ a $C$ se obtiene un conjunto $C' = {C + v}$ que es independiente, o sea que $C \subset C'$, lo cual es absurdo, puesto que habíamos dicho que $C$ era maximal. El absurdo proviene de suponer que el conjunto no es dominante, por lo tanto, el conjunto debe ser dominante.

\subsection{Todo conjunto independiente dominante es independiente maximal.}
Lo probaremos por absurdo.  Supongamos que existe un grafo $G = (V,E)$ tal que tiene un conjunto independiente dominante $I$ que no es independiente maximal. Como $I$ no es independiente maximal, podemos suponer que existe algún conjunto $J \subseteq V$ independiente, tal que $I \subset J$. Entonces, existe $v$ un nodo que pertenece a $J$ pero no a $I$, tal que $v$ no es vecino de ningún elemento de $I$, lo cual es absurdo, ya que suponer eso es decir que $I$ no era dominante. 

\subsection{Conclusión}
Por las propiedades anteriormente demostradas, se puede concluir que el problema de buscar un CIDM es idéntico al problema de buscar un conjunto independiente maximal mínimo(CIMM), o sea, de todos los conjuntos independientes maximales que son posibles formar dado un grafo cualquiera, aquel cuya cardinalidad es la menor. Por esta razón, nuestros algoritmos buscarán encontrar o aproximar un CIMM para un grafo dado.

\vspace*{0.6cm}

\section{Plataforma de pruebas}

\vspace*{0.3cm}

Para toda la experimentación se utilizará un procesador Intel Core i3, de 4 nucleos a 2.20 GHZ.

El software utilizado será Ubuntu 14.04, y G++ 4.8.2.

\vspace*{0.6cm}

\section{Algoritmo Exacto}
\vspace*{0.3cm}
\subsection{Desarrollo de la idea y correctitud.}

\vspace*{0.3cm}

Para resolver el problema de CIDM de manera exacta hemos decidido utilizar la técnica de backtracking. Por todo lo dicho en la seccion de propiedades, nuestro backtracking se encargará de, dado un grafo, ver todos los posibles conjuntos independientes maximales, y tomará aquel que sea menor en cardinalidad. 

Para esto, le daremos a los nodos un orden en particular, y para cada nodo consideraremos las siguientes dos opciones o ``ramas'':

\begin{itemize}
	\item Tomar el nodo como parte del conjunto solución. Esta rama sólo será considerada cuando el nodo no sea adyacente a otro nodo que ya fue colocado anteriormente en el conjunto.  Por eso, en caso de tomar el nodo actual, marcaremos a este nodo y a todos sus vecinos, de manera de no tomarlos nuevamente en el futuro de esa rama, puesto que si tomásemos a alguno de ellos en el conjunto, este no sería independiente. 
	\item No tomar el nodo como parte del conjunto solución.  En este caso no se hará nada, y se avanzará hacia el próximo nodo, de existir éste.
\end{itemize}

Luego de la elección tomada para un determinado nodo, consultaremos las siguientes posibilidades:

\begin{itemize}
	\item Si el nodo tratado en el último paso es el último nodo y no están todos los nodos marcados, el conjunto obtenido no es un independiente maximal, por lo que no lo tomamos en cuenta.
	\item Si todos los nodos quedaron marcados, el conjunto obtenido es independiente maximal.  Se verá entonces la cardinalidad de este conjunto, y de ser mejor que el de la mejor solución obtenida hasta el momento, se lo guardará como nueva mejor solución.
	\item De no haber visto el último nodo, y de existir nodos no marcados aún, avanzaremos al siguiente nodo y repetiremos el procedimiento.
\end{itemize}

Para poder afirmar que nuestro algoritmo es correcto, basta con poder probar que todo conjunto que forma es independiente maximal, y que realmente observa todo conjunto independiente maximal de un grafo:

\begin{itemize}
	\item Podemos afirmar que este procedimiento encuentra {\bf conjuntos independientes}, puesto que sólo se toman aquellos nodos que no están marcados, o sea, que no tienen ninguna arista en común con los elementos del conjunto.
	\item Podemos afirmar que este procedimiento encuentra conjuntos independientes que son {\bf maximales} puesto que el programa deja de agregar nodos cuando todos estos están marcados, lo que significa que todos los nodos del grafo, o bien son adyacentes a algún elemento del conjunto, o bien están dentro del conjunto. Por eso, no podemos tomar ningún nuevo elemento de modo tal que el nuevo conjunto sea independiente.
	\item Podemos afirmar que se observan {\bf todos} los posibles conjuntos independientes maximales por lo siguiente:  sea $C$ un conjunto independiente maximal del grafo $G$, conformado por los vertices $v_{1}, v_{2}, ... , v_{h}$, entonces, por cómo esta diseñado nuestro algoritmo, se llegaría a observar este conjunto independiente maximal cuando estemos en la rama que solo toma a $v_{1}, v_{2}, ... , v_{h}$ y no toma a los demás. Notemos que esta rama existe, pues $v_{1}, v_{2}, ... , v_{h}$ son nodos independientes, y por lo tanto, al tomar uno de ellos, los otros no se marcan y quedan disponibles para ser tomados como parte de la solución.
\end{itemize}

Para mejorar la velocidad de ejecución del algoritmo, se han aplicado las siguientes podas:

\begin{itemize}
	\item Poda Clásica: Si en la rama actual que está revisando nuestro algoritmo, la cantidad de elementos del conjunto maximal de esta rama es mayor a la cantidad de elementos de la mejor solución encontrada hasta el momento, esta rama deja de ser considerada, puesto que, de conseguir una solución, seguro no es la mejor.
	\item Nodos solitarios: Si el grafo tiene algún nodo con grado 0, no tiene sentido considerar la opción de no tomarlo como parte de la solución, por lo cual dicha rama no será revisada cuando el nodo cumple esta característica.
	\item El nodo decisivo: Si durante el procesamiento de un nodo, al tomarlo, cubre a todos los que estaban libres hasta el momento, entonces no se considerará la rama resultante de no tomarlo, ya que en el mejor de los casos uno de los nodos siguientes también cubre a todos y esto no mejora la cardinalidad del conjunto hallado, y en casos peores, será necesario tomar más de uno de los nodos siguientes para lograr un conjunto independiente maximal.
\end{itemize}

\vspace*{0.6cm}

%\newpage

\subsection{Análisis de complejidad.}

\vspace*{0.3cm}

\begin{figure}
\begin{codebox}
\Procname{$\proc{CIDM_exacto}(lista\_nodos$ $cidm,lista\_nodos$ $cidm\_sol,nodo,int$ $n,int$ $cota,int$ $res)$} 
\li \If se superó la $cota$
\li \Then \Return
	\End
\li \If se encontró una solución
\li \Then 
 		$cidm\_sol \leftarrow cidm$
\li 		$cota \leftarrow res$
\li 		\Return
	\End
\li \If se llegó al final y no se encontró una solución
\li \Then \Return
	\End
\li \If $nodo$ no está ``tomado''	
\li \Then
		$cidm \leftarrow$ agregar $nodo$
\li 		incrementar $res$
\li 		marcar a $nodo$ y a sus vecinos como ``tomados''
\li		{\sc CIDM_exacto}($cidm,cidm\_sol,nodo\_siguiente,n,cota,res$)
	\End
\li \If $nodo$ tiene vecinos ó todavía no se llegó a una solución
\li \Then
 		\If se modificó en la rama anterior
\li 		\Then
			$cidm \leftarrow$ sacar $nodo$
\li			decrementar $res$
\li 			marcar a $nodo$ y a sus vecinos como ``no tomados''
		\End
\li 		{\sc CIDM_exacto}($cidm,cidm\_sol,nodo\_siguiente,n,cota,res$)
\end{codebox}
\caption{Algoritmo exacto para CIDM}\label{code:exacto}
\end{figure}
%\FloatBarrier


\vspace*{0.6cm}
%\newpage
\subsection{Experimentación y gráficos.}

\vspace*{0.3cm}


\subsubsection{Test 1}
\vspace*{0.3cm}

\vspace*{0.6cm}
%\newpage

\subsubsection{Test 2}



%\newpage
\vspace*{0.6cm}
\section{Heurística Golosa Constructiva}
\vspace*{0.3cm}
\subsection{Desarrollo de la idea.}

\vspace*{0.3cm}

Sea $G$ un grafo cualquiera, $I$ un conjunto independiente de ese nodo, y $n_{1}$,$n_{2}$ nodos de G que no pertenecen a $I$ y no tienen aristas en común con ningún elemento de $I$, diremos que $n_{1}$ es óptimo si no existe ningún $n_{2}$ tal que $( \# (Vecinos(n_{2})) - \# (Vecinos(n_{2}) \cap Vecinos(I))) > ( \# (Vecinos(n_{1})) - \# (Vecinos(n_{1}) \cap Vecinos(I)))$. Definimos $Vecinos(\alpha)$ como el conjunto de nodos a los cuales $\alpha$ lleva una arista si $\alpha$ es un nodo, y si $\alpha$ es un conjunto, entonces es el conjunto de nodos a los cuales les llega un arista desde por lo menos un elemento de $\alpha$.  De manera más informal, podemos decir que un nodo óptimo es aquél que más vecinos ``libres'' tiene, siendo un nodo ``libre'' uno que no está en el conjunto solución ni es adyacente a un nodo de la solución.

Nuestra heurística se basa en formar un conjunto independiente maximal de la siguiente manera:
Primero, considerando al conjunto independiente vacío, tomamos a un nodo óptimo, y lo agregamos como nuevo elemento de nuestro conjunto independiente. Luego con nuestro nuevo conjunto independiente, tomamos un nodo óptimo, y lo agregamos a nuestro conjunto independiente. Repetimos esto hasta que no exista un nodo óptimo, puesto que nuestro conjunto independiente se transformó en maximal, y por lo tanto en un conjunto dominante.
 
\vspace*{0.6cm}

%\newpage

\subsection{Análisis de complejidad.}

\vspace*{0.3cm}

\begin{figure}
\begin{codebox}
\Procname{$\proc{CIDM_goloso}(lista\_nodos$ $cidm\_sol)$} 
\li $res \leftarrow$ 0
\li \While no se hayan ``tomado'' todos los nodos
\li 	\Do 
 		$elegido \leftarrow$ nodo ``óptimo''
\li 		$cidm\_sol \leftarrow$ agregar $elegido$
\li 		incrementar $res$
\li 		marcar a $elegido$ y a sus vecinos como ``tomados''
	\End
\li \Return $res$
\end{codebox}
\caption{Heurística golosa constructiva para CIDM}\label{code:goloso}
\end{figure}
%\FloatBarrier

\vspace*{0.6cm}
%\newpage

\subsection{Instancias no óptimas.}


\subsection{Experimentación y gráficos.}

\vspace*{0.3cm}


\subsubsection{Test 1}
\vspace*{0.3cm}

\vspace*{0.6cm}
%\newpage

\subsubsection{Test 2}



\newpage
\section{Heurística de Busqueda Local}
\subsection{Desarrollo de la idea y correctitud.}

\vspace*{0.3cm}


 
\vspace*{0.6cm}

%\newpage

\subsection{Análisis de complejidad de una iteración.}

\vspace*{0.3cm}


\vspace*{0.6cm}
%\newpage
\subsection{Experimentación y gráficos.}

\vspace*{0.3cm}


\subsubsection{Test 1}
\vspace*{0.3cm}

\vspace*{0.6cm}
%\newpage

\subsubsection{Test 2}



\newpage
\section{Metaheurística de GRASP}
\subsection{Desarrollo de la idea.}

\vspace*{0.3cm}

GRASP (Greedy Randomized Adaptative Search Procedure) es una combinación entre una heurística golosa ``aleatorizada'' y un procedimiento de búsqueda local.  La idea es la siguiente:

\begin{codebox}
\li \While no se alcance el {\it criterio de terminación}
\li \Do 
		Obtener una solución inicial mediante una {\it heurística golosa aleatorizada}.
\li 		Mejorar la solución mediante búsqueda local.
\li 		Recordar la mejor solución obtenida hasta el momento.
	\End
\end{codebox}

Como criterios de terminación, analizaremos dos casos particulares:

\begin{itemize}
\item Se realizaron $k$ iteraciones.
\item Se realizaron $k$ iteraciones sin encontrar una solución mejor.
\end{itemize}

En cuanto a la heurística golosa aleatorizada, hemos continuado con la idea del algoritmo goloso descrito anteriormente, con la variación de que, en cada paso, se genera una Lista Restricta de Candidatos (RCL) y se elige aleatoriamente un candidato de esa lista.  Analizaremos dos posibles maneras de construir dicha RCL:

\begin{itemize}
\item Hallar al mejor candidato, es decir, el nodo que hemos definido como óptimo, y colocar en la RCL los nodos candidatos cuya cantidad de vecinos ``libres'' sea no menor a un cierto porcentaje $\alpha$ de la cantidad de vecinos ``libres'' del mejor candidato.
\item Hallar al mejor candidato y colocar en la RCL los $\beta$ mejores candidatos, es decir, los $\beta$ nodos que más nodos ``libres'' cubran.
\end{itemize}

\vspace*{0.6cm}

%\newpage
\subsection{Experimentación y gráficos.}

\vspace*{0.3cm}


\subsubsection{Test 1}
\vspace*{0.3cm}

\vspace*{0.6cm}
%\newpage

\subsubsection{Test 2}



%\newpage
%\section{Comparación de los distintos métodos}
%\subsection{Experimentación y gráficos.}

\vspace*{0.3cm}


\subsubsection{Test 1}
\vspace*{0.3cm}

\vspace*{0.6cm}
%\newpage

\subsubsection{Test 2}


%
%\newpage
%\section{Apéndice 1: acerca de los tests}
%
%
%\subsection{Código del Problema 1}
%
%%\newpage
%\subsection{Código del Problema 2}
%
%
%
%%\newpage
%\subsection{Código del Problema 3}
%

\end{document}

\documentclass[a4paper]{article}
\usepackage[spanish]{babel}
\usepackage[utf8]{inputenc}
\usepackage{fancyhdr}
\usepackage{charter}   % tipografía
\usepackage{graphicx}
\usepackage{makeidx}

\usepackage{float}
\usepackage{amsmath, amsthm, amssymb}
\usepackage{amsfonts}
\usepackage{sectsty}
\usepackage{wrapfig}
\usepackage{listings} % necesario para el resaltado de sintaxis
\usepackage{caption}
\usepackage{placeins}

\usepackage{hyperref} % agrega hipervínculos en cada entrada del índice
\hypersetup{          % (en el pdf)
    colorlinks=true,
    linktoc=all,
    citecolor=black,
    filecolor=black,
    linkcolor=black,
    urlcolor=black
}

\usepackage{color} % para snippets de código coloreados
\usepackage{fancybox}  % para el sbox de los snippets de código

\definecolor{litegrey}{gray}{0.94}

% \newenvironment{sidebar}{%
% 	\begin{Sbox}\begin{minipage}{.85\textwidth}}%
% 	{\end{minipage}\end{Sbox}%
% 		\begin{center}\setlength{\fboxsep}{6pt}%
% 		\shadowbox{\TheSbox}\end{center}}
% \newenvironment{warning}{%
% 	\begin{Sbox}\begin{minipage}{.85\textwidth}\sffamily\lite\small\RaggedRight}%
% 	{\end{minipage}\end{Sbox}%
% 		\begin{center}\setlength{\fboxsep}{6pt}%
% 		\colorbox{litegrey}{\TheSbox}\end{center}}

\newenvironment{codesnippet}{%
	\begin{Sbox}\begin{minipage}{\textwidth}\sffamily\small}%
	{\end{minipage}\end{Sbox}%
		\begin{center}%
		\colorbox{litegrey}{\TheSbox}\end{center}}



\usepackage{fancyhdr}
\pagestyle{fancy}

%\renewcommand{\chaptermark}[1]{\markboth{#1}{}}
\renewcommand{\sectionmark}[1]{\markright{\thesection\ - #1}}

\fancyhf{}

\fancyhead[LO]{Sección \rightmark} % \thesection\
\fancyfoot[LO]{\small{integrante 1, integrante 2, integrante 3, ..., integrante n}}
\fancyfoot[RO]{\thepage}
\renewcommand{\headrulewidth}{0.5pt}
\renewcommand{\footrulewidth}{0.5pt}
\setlength{\hoffset}{-0.8in}
\setlength{\textwidth}{16cm}
%\setlength{\hoffset}{-1.1cm}
%\setlength{\textwidth}{16cm}
\setlength{\headsep}{0.5cm}
\setlength{\textheight}{25cm}
\setlength{\voffset}{-0.7in}
\setlength{\headwidth}{\textwidth}
\setlength{\headheight}{13.1pt}

\renewcommand{\baselinestretch}{1.1}  % line spacing


\usepackage{underscore}
\usepackage{caratula}
\usepackage{url}
\usepackage{color}
\usepackage{clrscode3e} % necesario para el pseudocodigo (estilo Cormen)




\begin{document}

\lstset{
  language=C++,                    % (cambiar al lenguaje correspondiente)
  backgroundcolor=\color{white},   % choose the background color
  basicstyle=\footnotesize,        % size of fonts used for the code
  breaklines=true,                 % automatic line breaking only at whitespace
  captionpos=b,                    % sets the caption-position to bottom
  commentstyle=\color{red},    % comment style
  escapeinside={\%*}{*)},          % if you want to add LaTeX within your code
  keywordstyle=\color{blue},       % keyword style
  stringstyle=\color{blue},     % string literal style
}

\thispagestyle{empty}
\materia{Algoritmos y Estructuras de Datos III}
\submateria{Primer Cuatrimestre de 2015}
\titulo{Diseño de Algoritmos}
\subtitulo{Aplicación de técnicas}
\integrante{Barañao, Facundo}{480/11}{facundo_732@hotmail.com}
\integrante{Confalonieri, Gisela Belén}{511/11}{gise_5291@yahoo.com.ar} % por cada integrante (apellido, nombre) (n° libreta) (e-mail)
\integrante{Mignanelli, Alejandro Rubén}{609/11}{minga_titere@hotmail.com} 
\integrante{Soliz, Carlos}{406/12}{rcarlos.cs@gmail.com} 

\maketitle
\newpage

\thispagestyle{empty}
\vfill
%\begin{abstract}
%    \vspace{0.5cm}
%	Este trabajo busca aplicar distintas técnicas para la creación de algoritmos. Para esto, se resolverán tres problemas que representan situaciones de la vida real y juegos de ingenio. Las técnicas que se utilizarán serán algoritmos golosos, divide and conquer y backtracking.
%
%\end{abstract}

\thispagestyle{empty}
\vspace{1.5cm}
\tableofcontents
\newpage

%\normalsize

\newpage
\section{Objetivos generales}

El objetivo principal del siguiente trabajo será la práctica de distintas técnicas para la creación de algoritmos. Dados tres problemas, que representan situaciones factibles de la vida real y juegos de ingenio, se buscará resolverlos de forma óptima mediante la aplicación de algoritmos golosos, divide and conquer y backtracking.  En algunos de los casos, se buscará que el algoritmo no supere cierta cota de complejidad en peor caso.  %Pretendemos mostrar cómo estas técnicas permiten mejorar los algoritmos frente a abordajes de ``fuerza bruta''.

%\newpage
\section{Plataforma de pruebas}

Para toda la experimentación se utilizará un procesador Intel Atom, de 4 núcleos a 1.6 GHZ.

El software utilizado será Ubuntu 14.04, y G++ 4.8.2.

\newpage
\section{Problema 1: Zombieland}
\subsection{Descripción del problema.}

\vspace*{0.3cm}

\textbf{completar!}


%\begin{figure}[htb]
%  \begin{center}
%      \includegraphics[scale=0.25]{imagenes/ejemplo.jpg}
%  \end{center}
%  \caption{ejemplo}
%\end{figure}



\newpage
\subsection{Desarrollo de la idea y pseudocódigo.}

\vspace*{0.3cm}

\textbf{completar!}

\begin{codebox}
\Procname{$\proc{ejemplo_de_pseudocodigo}(x,y)$}
\li \Return $\id{solucion}$
\end{codebox}



\newpage
\subsection{Justificación de la resolución y demostración de correctitud.}

\vspace*{0.3cm}

\textbf{completar!}



\newpage
\subsection{Análisis de complejidad.}

\vspace*{0.3cm}

\textbf{completar!}



\newpage
\subsection{Experimentación y gráficos.}

\vspace*{0.3cm}

\subsubsection{Test 1}

\vspace*{0.3cm}

\textbf{completar!}


\newpage
\subsubsection{Test 2}

\vspace*{0.3cm}

\textbf{completar!}


\newpage
\subsubsection{Test 3}

\vspace*{0.3cm}

\textbf{completar!}


\newpage
\section{Problema 2: Alta Frecuencia}
\subsection{Descripción del problema.}

\vspace*{0.3cm}

Luego del éxito rotundo del último ataque contra los zombies producido por nuestro anterior algoritmo, la humanidad ve un rayo de esperanza.
En una de las últimas ciudades tomadas por la amenaza Z, se encuentra un científico que afirma haber encontrado la cura contra el zombieismo, rodeado por una cuadrilla de soldados del más alto nivel. Nuestro noble objetivo es entonces, proveer del camino más seguro(el que genere menos perdidas humanas) al científico y sus soldados de manera tal que logren llegar desde donde están, hasta el bunker militar que se encuentra en esa ciudad. Para esto, se conocen los siguientes datos:
\begin{itemize}

	\item La ciudad en cuestión tiene la forma clásica de grilla rectangular, compuesta por $n$ calles paralelas en forma horizontal, $m$ calles paralelas en forma vertical dando así una grilla de manzanas cuadradas. Los numeros $n$ y $m$ son conocidos.
	\item Se conoce la ubicación del científico y del bunker.
	\item Se conoce la cantidad de soldados que el científico tiene a su disposición.
	\item Si todos los soldados perecen, el científico no tiene posibilidad de sobrevivir por si mismo(o sea, no puedo llegar al bunker con 0 soldados).
	\item Se sabe la cantidad de zombies ubicados en cada calle.
	\item Si bien nuestros soldados tienen un alto nivel de combate, el enfrentamiento que tuvieron en el último ataque contra los zombies, acabaron con todas sus municiones por lo cual solo tienen sus cuchillos de combate. Esto incide entonces, en que el grupo solo pasara por una calle sin sufrir perdidas, si hay hasta un soldado por zombie, al menos, y en caso contrario, perdera la diferencia entre la cantidad de soldados, y la cantidad de zombies. En otras palabras, sea $z$ la cantidad de zombies de una calle, y $s$ la cantidad de soldados de que tiene la cuadrilla al momento de pasar por esa calle, si $s \geq z$ entonces la cuadrilla pasa sin perdidas. En caso contrario la cuadrilla pierde $z - s$ soldados.
	\item Puede no existir un camino que asegure la supervivencia del científico.
	\item La complejidad del algoritmo pedido es de $\mathcal{O}(s \cdot n \cdot m)$.


\end{itemize}

Ejemplo: 

PENSAR EJEMPLO, Y MOSTRAR LA ENTRADA CON UN DIBUJITO, ES MÁS ENTENDIBLE (ALE)

\vspace*{0.6cm}

%\newpage
\subsection{Desarrollo de la idea y correctitud.}

\vspace*{0.3cm}


\vspace*{0.6cm}

%\newpage
%\subsection{Justificación de la resolución y demostración de correctitud.}

%\vspace*{0.3cm}



%\vspace*{0.6cm}

%\newpage
\subsection{Análisis de complejidad.}

\vspace*{0.3cm}


\vspace*{0.6cm}

%\newpage
\subsection{Experimentación y gráficos.}


\subsubsection{Test 1}

\vspace*{0.3cm}


\vspace*{0.6cm}

%\newpage
\subsubsection{Test 2}


\newpage

\newpage
\section{Problema 3: El señor de los caballos}
\subsection{Descripción del problema.}

\vspace*{0.3cm}

\textbf{completar!}


%\begin{figure}[htb]
%  \begin{center}
%      \includegraphics[scale=0.25]{imagenes/ejemplo.jpg}
%  \end{center}
%  \caption{ejemplo}
%\end{figure}



\newpage
\subsection{Desarrollo de la idea y pseudocódigo.}

\vspace*{0.3cm}

\textbf{completar!}

\begin{codebox}
\Procname{$\proc{ejemplo_de_pseudocodigo}(x,y)$}
\li \Return $\id{solucion}$
\end{codebox}



\newpage
\subsection{Justificación de la resolución y demostración de correctitud.}

\vspace*{0.3cm}

\textbf{completar!}



\newpage
\subsection{Análisis de complejidad.}

\vspace*{0.3cm}

\textbf{completar!}



\newpage
\subsection{Experimentación y gráficos.}

\vspace*{0.3cm}

\subsubsection{Test 1}

\vspace*{0.3cm}

\textbf{completar!}


\newpage
\subsubsection{Test 2}

\vspace*{0.3cm}

\textbf{completar!}


\newpage
\subsubsection{Test 3}

\vspace*{0.3cm}

\textbf{completar!}



\newpage
\section{Apéndice 1: acerca de los tests}

Para reproducir los tests mencionados en el presente informe, se adjunta con la entrega el script {\it tests_nuestros.sh} y una carpeta llamada {\it tests_caballos} con las instancias utilizadas para las pruebas del tercer problema.  Además, se provee el script {\it compilar.sh} para compilar los programas, y el archivo {\it testing.cpp} que se ocupa de correr los tests para los primeros dos problemas.  Estos dos últimos se llaman desde {\it tests_nuestros.sh}.

El script mencionado se ocupa de realizar las compilaciones necesarias y correr cada una de las pruebas. Los resultados se guardarán en un directorio llamado {\it Resultados_tests_nuestros}.

Para que los tests correspondientes al problema 3 corran apropiadamente, se deben descomentar las líneas 141 y 209 a 212 del archivo {\it caballos.cpp} .

El formato de la salida de cada test será el siguiente:

\begin{itemize}
\item {\bf Problema 1:} Una línea por cada instancia evaluada, cada una de las cuales estará formada por un entero que representa la cantidad de ciudades, un entero que representa el presupuesto utilizado, y un float que representa el tiempo incurrido (en ciclos de clock).
\item {\bf Problema 2:} Una línea por cada instancia evaluada, cada una de las cuales estará formada por un entero que representa la cantidad de frecuencias, un caracter 'p', 't' o 'c' para identificar los casos ``pirámide'', ``tren'' y ``cadena'', y un float que representa el tiempo incurrido (en ciclos de clock).
\item {\bf Problema 3:} Cincuenta líneas por cada instancia evaluada (en este caso el promedio no se calcula de forma automática), cada una de las cuales estará formada por un entero que representa la cantidad de casilleros por lado, un entero que representa la cantidad de caballos preubicados, y un float que representa el tiempo incurrido (en ciclos de clock).
\end{itemize}

\newpage
\section{Apéndice 2: secciones relevantes del código}
En esta sección, adjuntamos parte del código correspondiente a la resolución de cada problema que consideramos más \textbf{relevante}.

\subsection{Código del Problema 1}
Estructura para ciudad:
\begin{lstlisting}
typedef struct ciudad_t{
	int nombre; 			// numero que identifica la ciudad segun el orden en el que viene dada
    int cant_zombies;
    int cant_soldados;
    int soldier_req;  	// numero de soldados que faltan para salvar la ciudad
    int costfsoldier;  	// costo de enviar un soldado
    int costfsafety;  	// costo de salvar esa ciudad
    bool salvar;			// 1 para salvarla, 0 para no salvarla
} ciudad;
\end{lstlisting}

\vspace*{0.5cm}

Funciones de comparación utilizadas para el list.sort:
\begin{lstlisting}
bool compare_cost(const ciudad& city1, const ciudad& city2){
	return (city1.costfsafety < city2.costfsafety);
}

bool compare_name(const ciudad& city1, const ciudad& city2){
	return (city1.nombre < city2.nombre);
}
\end{lstlisting}

\vspace*{0.5cm}

Función golosa:
\begin{lstlisting}
int zombie_goloso(list<ciudad>& city, int p){
/* Devuelve la cantidad de ciudades que se salvan */

	int i = 0;
	int sum = 0;
	list<ciudad>::iterator it = city.begin();
    
    while(it != city.end() && (sum < p)){
		sum = sum + it->costfsafety;
		if (sum <= p){
			it->salvar = 1;
			i++;
		}
		it++;
	}
	return i;
}
\end{lstlisting}

%\newpage
\subsection{Código del Problema 2}
Estructura para frecuencia:
\begin{lstlisting}
typedef struct frecuencia_t{
	int nombre; 			// numero que identifica la frecuencia segun el orden en el que viene dada
	int costfminute;  	// costo de uso por minuto
    int inicio;			// minuto inicio de transmision
    int fin;				// minuto fin de transmision
} frecuencia;
\end{lstlisting}

\vspace*{0.5cm}

Función de comparación utilizada para el list.sort:
\begin{lstlisting}
bool compare_time(const frecuencia& freq1, const frecuencia& freq2){
	return (freq1.inicio < freq2.inicio);
}
\end{lstlisting}

Función ocupada del divide and conquer:
\begin{lstlisting}
list<frecuencia> frequency_dc(frecuencia* freq, int low, int high)
{
	int mid;
	list<frecuencia> trans1;
	list<frecuencia> trans2;
	list<frecuencia> trans_final;
	
    if (low < high)	// Divido en 2 y hago recursion
    {
        mid = (low+high)/2;
        trans1 = frequency_dc(freq,low,mid);
        trans2 = frequency_dc(freq,mid+1,high);
		trans_final = mezclar_freq(trans1,trans2);
    }else{	// Es un solo elemento, hago una lista con esa transmision
		trans_final.push_back(freq[low]);
	}
	trans1.clear();
	trans2.clear();
    return trans_final;
}
\end{lstlisting}

%\newpage
\subsection{Código del Problema 3}
Estrucura para Tablero:
\begin{lstlisting}
typedef vector<char> Vec;
typedef vector<Vec> Tablero;
\end{lstlisting}


Función ocupada de realizar el backtraking:
\begin{lstlisting}
void backtranki(Tablero& tab,Tablero& tab_final, int fila, int columna, int n, int& cota, int lo_que_falta, int extras)
{
	int f;
	int c;
	
	if(extras > cota)
	{	// Si me pase de la cota, me voy.
		return;
	}

	if(lo_que_falta == 0)	// Si ya llene el tablero, me guardo esa solucion y me voy
	{
		copiar_tablero(tab,tab_final,n);
		cota = extras;
		return;
	}
	if(fila == n){		// Si ya recorri todo el tablero, me voy
		return;
	}
	
	// Establezco la fila y columna para la siguiente llamada recursiva
	f = fila;
	c = columna + 1;
	if(c == n){
		f = fila +1;
		c = 0;
	}
	
	if(tab[fila][columna] != 'p')	// Si no habia un caballo preubicado
	{
		Tablero copia_tab(n, Vec(n, 'v'));
		copiar_tablero(tab,copia_tab,n);	// Hago una copia del tablero

		int r = lo_que_falta;				//lo pongo en variable aparte, para que no rompa a
		int e = extras;						//lo_que_falta y a extras que estan fuera del if
		
		if(copia_tab[fila][columna] == 'v'){

			r--;
		}

		copia_tab[fila][columna] = 'e';	// pongo un caballo extra
		e++;
		
		// actualizo el tablero y lo que falta
		setear_amenaza(copia_tab,fila-2,columna-1,n,r);	
		setear_amenaza(copia_tab,fila-2,columna+1,n,r);
		setear_amenaza(copia_tab,fila-1,columna-2,n,r);	
		setear_amenaza(copia_tab,fila-1,columna+2,n,r);
		setear_amenaza(copia_tab,fila+1,columna-2,n,r);
		setear_amenaza(copia_tab,fila+1,columna+2,n,r);
		setear_amenaza(copia_tab,fila+2,columna-1,n,r);
		setear_amenaza(copia_tab,fila+2,columna+1,n,r);
		
		// llamada recursiva con el tablero actualizado
		backtranki(copia_tab,tab_final,f,c,n,cota,r,e);
	}

	// Llamada recursiva con el mismo tablero
	backtranki(tab,tab_final,f,c,n,cota,lo_que_falta,extras);
}
\end{lstlisting}

\end{document}

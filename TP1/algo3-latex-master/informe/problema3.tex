\subsection{Descripción del problema.}

\vspace*{0.3cm}

\textbf{Se tiene un juego de mesa cuyo tablero, dividido en casillas, posee igual cantidad de filas y columnas y hace uso de una conocida pieza del popular ajedrez: el caballo. El juego es solamente para un jugador y consiste en, teniendo caballos ubicados en distintos casilleros, insertar en casilleros vacíos la mínima cantidad de caballos extras, de manera tal que, siguendo las reglas del movimiento de los caballos en el ajedrez, todas las casillas se encuentren ocupadas o amenazadas por un caballo.
Aspectos a tener en cuenta:
}


%\begin{figure}[htb]
%  \begin{center}
%      \includegraphics[scale=0.25]{imagenes/ejemplo.jpg}
%  \end{center}
%  \caption{ejemplo}
%\end{figure}



\newpage
\subsection{Desarrollo de la idea y pseudocódigo.}

\vspace*{0.3cm}

\textbf{completar!}

\begin{codebox}
\Procname{$\proc{ejemplo_de_pseudocodigo}(x,y)$}
\li \Return $\id{solucion}$
\end{codebox}



\newpage
\subsection{Justificación de la resolución y demostración de correctitud.}

\vspace*{0.3cm}

\textbf{completar!}



\newpage
\subsection{Análisis de complejidad.}

\vspace*{0.3cm}

\textbf{completar!}



\newpage
\subsection{Experimentación y gráficos.}

\vspace*{0.3cm}

\subsubsection{Test 1}

\vspace*{0.3cm}

\textbf{completar!}


\newpage
\subsubsection{Test 2}

\vspace*{0.3cm}

\textbf{completar!}


\newpage
\subsubsection{Test 3}

\vspace*{0.3cm}

\textbf{completar!}

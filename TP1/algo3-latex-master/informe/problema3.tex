\subsection{Descripción del problema.}

\vspace*{0.3cm}

Se tiene un juego de mesa cuyo tablero, dividido en casillas, posee igual cantidad de filas y columnas y hace uso de una conocida pieza del popular ajedrez: el caballo. El juego es solamente para un jugador y consiste en, teniendo caballos ubicados en distintos casilleros, insertar en casilleros vacíos la mínima cantidad de caballos extras, de manera tal que, siguendo las reglas del movimiento de los caballos en el ajedrez, todas las casillas se encuentren ocupadas o amenazadas por un caballo.
Aspectos a tener en cuenta:

\begin{itemize}
   \item Se conoce la cantidad de filas y columnas del tablero
   \item Se conoce la cantidad de caballos que ocupan el tablero inicialmente.
   \item Para cada uno de estos caballos, se sabe su ubicación en el tablero
   \item Una casilla se considera amenazada si existe un caballo tal que en una movida pueda ocupar dicha casilla
\end{itemize}

%\begin{figure}[htb]
%  \begin{center}
%      \includegraphics[scale=0.25]{imagenes/ejemplo.jpg}
%  \end{center}
%  \caption{ejemplo}
%\end{figure}

\vspace*{0.6cm}

%\newpage
\subsection{Desarrollo de la idea y correctitud.}


\begin{figure}[!htb]
\minipage{0.32\textwidth}
  \includegraphics[scale=1]{imagenes/tab1.png}
  \caption{Tablero 1}\label{fig:tab1}
\endminipage\hfill
\minipage{0.32\textwidth}
  \includegraphics[scale=1]{imagenes/tab2.png}
  \caption{Tablero 2}\label{fig:tab2}
\endminipage\hfill
\minipage{0.32\textwidth}%
  \includegraphics[scale=1]{imagenes/tab3.png}
  \caption{Tablero 3}\label{fig:tab3}
\endminipage
\end{figure}

%\begin{figure}[htb]
 % \begin{center}
%     \includegraphics[scale=1]{imagenes/tab1.png}
%  \end{center}
 % \caption{Tablero 1}\label{fig:tab1}
%\end{figure}
%\vspace*{0.3cm}

%\begin{figure}[htb]
 % \begin{center}
  %    \includegraphics[scale=1]{imagenes/tab2.png}
  %\end{center}
  %\caption{Tablero 2}\label{fig:tab2}
%\end{figure}

%\begin{figure}[htb]
 % \begin{center}
  %    \includegraphics[scale=1]{imagenes/tab3.png}
  %\end{center}
  %\caption{Tablero 3}\label{fig:tab3}
%\end{figure}

Supongamos un ejemplo como el del Figura \ref{fig:tab1}, donde 'v' indica casillero vacío y 'p' un caballo preubicado. Observemos que tanto Figura \ref{fig:tab2} y Figura \ref{fig:tab3} son soluciones (donde 'e' indica que se agregó un caballo extra y 'a' que dicha casilla esta amenazada sin un caballo colocado). Esto significa que en principio, puede haber múltiples soluciones óptimas. Más aún, veamos que si el tablero fuera de 2x2, la única manera de obtener un tablero completo (diremos que un tablero está completo si todas sus casillas, o tienen un caballo, o están amenazadas por uno) es que haya un caballo en cada casilla, o sea que puedo tener soluciones óptimas como en 3x3 o puedo necesitar completar el tablero con caballos como en 2x2. 

Veamos qué sucede para 4x4.


\begin{figure}[!htb]
\minipage{0.32\textwidth}
  \includegraphics[scale=1]{imagenes/tab4.png}
  \caption{Tablero 4}\label{fig:tab4}
\endminipage\hfill
\minipage{0.32\textwidth}
  \includegraphics[scale=1]{imagenes/tab5.png}
  \caption{Tablero 5}\label{fig:tab5}
\endminipage\hfill
\minipage{0.32\textwidth}%
  \includegraphics[scale=1]{imagenes/tab6.png}
  \caption{Tablero 6}\label{fig:tab6}
\endminipage
\end{figure}

%\begin{figure}[htb]
% \begin{center}
%    \includegraphics[scale=1]{imagenes/tab4.png}
%  \end{center}
%  \caption{Tablero 4}\label{fig:tab4}
%\end{figure}

%\begin{figure}[htb]
%  \begin{center}
%      \includegraphics[scale=1]{imagenes/tab5.png}
%  \end{center}
%  \caption{Tablero 5}\label{fig:tab5}
%\end{figure}

%\begin{figure}[htb]
%  \begin{center}
%      \includegraphics[scale=1]{imagenes/tab6.png}
%  \end{center}
%  \caption{Tablero 6}\label{fig:tab6}
%\end{figure}

\begin{figure}[!htb]
\minipage{0.32\textwidth}
  \includegraphics[scale=1]{imagenes/tab7.png}
  \caption{Tablero 7}\label{fig:tab7}
\endminipage
\minipage{0.32\textwidth}
  \includegraphics[scale=1]{imagenes/tab8.png}
  \caption{Tablero 8}\label{fig:tab8}
\endminipage
\end{figure}

%\begin{figure}[htb]
%  \begin{center}
%      \includegraphics[scale=1]{imagenes/tab7.png}
%  \end{center}
%  \caption{Tablero 7}\label{fig:tab7}
%\end{figure}

%\begin{figure}[htb]
%  \begin{center}
%      \includegraphics[scale=1]{imagenes/tab8.png}
%  \end{center}
%  \caption{Tablero 8}\label{fig:tab8}
%\end{figure}

Notemos que la Figura \ref{fig:tab4} es un ejemplo en el que pudimos insertar 2 caballos, que tanto la Figura \ref{fig:tab5} como la Figura \ref{fig:tab6} son solución de un mismo caso inicial (el tablero que tiene solo los caballos preubicados), y que en el Figura \ref{fig:tab7} no es necesario agregar ningún caballo extra. Esto significa que hay casos muy variados y amplios, lo cual vuelve difícil construir un algoritmo que resuelva el problema. Ante esto, como somos malos perdedores, decidimos sacrificar tiempo, pero vamos a encontrar una solución óptima.

Podemos afrontar este problema aplicando la técnica de backtracking, planteando un algoritmo que chequee todas las posibles formas de insertar caballos en un determinado tablero, desde rellenar un tablero con caballos hasta no poner ninguno, y se quede con una solución tal que el tablero esté completo y la cantidad de caballos insertados sea menor o igual al de todo el resto de las soluciones.

Una idea más ordenada de este algoritmo de backtracking es la siguiente (suponiendo que el tablero de entrada no está completo, puesto que sino se devolverá este mismo tablero):
\begin{itemize}
\item Recorriendo el tablero: empezamos por el casillero de la primera fila y columna para ir avanzando por los distintos casilleros de esa fila, en orden, y cuando se llegue al final de una fila, continuaremos por el primer casillero de la fila siguiente. De esta manera, recorremos el tablero completo, asegurandonos que se pasa por todas las casillas (A).
\item Insertando caballos: para cada casillero, si este tiene un caballo "p" colocado, se verá el siguiente casillero, y de no existir, la solución dada hasta el momento se tomará como la óptima. Sino, abordaremos dos posibilidades. Primero veremos el caso en el que se coloca un caballo en ese casillero, y después veremos el caso en el que no se agrega dicho caballo \ref{fig:diag}. En cada caso, se verá si el tablero queda o no completo, y si queda completo, se lo comparará con el resto de las soluciones completas, para ver si la cantidad de caballos agregados es menor o igual a todas o no. En caso de serlo, se la considerará la solución hasta que se encuentre otra mejor. De no encontrarse una solución mejor en los siguientes casos que se observen, será considerada mi solución óptima. Junto con (A), esto significa que se ven todas las posibles ramas de decisión \ref{fig:diag}, y por cada decisión se pregunta si es una solución, y si lo es, si es mejor que las anteriores, lo que quiere decir que pregunta por todas las soluciones y se queda con la mejor efectivamente.
\end{itemize}
\begin{figure}[htb]
  \begin{center}
      \includegraphics[scale=0.6]{imagenes/tab9.png}
  \end{center}
  \caption{Diagrama de acción}\label{fig:diag}
\end{figure}

Teniendo en cuenta estos métodos, llegamos a la conclusión de que nuestro algoritmo es correcto, sin embargo, ver todos los casos resulta tedioso y si bien tenemos tiempo, los docentes de algo 3 se molestarían si perdemos demasiado, lo cual motiva a aplicar cotas estratégicas. Estas funcionarán de la siguiente manera, si un caso dado supera a la cota, se desechará, puesto que ese camino seguro no llega a una solución óptima. De encontrarse una solución que mejora a la cota, entonces esta solución pasará a ser la nueva cota.

Analicemos qué cotas tomar:

No podemos pensar un caso por cada tamaño de tablero, puesto que no terminaríamos nunca, pero sí podemos pensar que si vemos el Tablero de la Figura \ref{fig:tab1} y reemplazamos la 'p' por una 'v', la solución Tablero \ref{fig:tab2} es óptima (cambiando su 'p' por una 'e') lo cual significa que, independientemente de los caballos 'p', con a lo sumo 4 se debe completar un tablero. Por otro lado, como mencionamos previamente, a un tablero de 2x2 hay que llenarlo de caballos, y si vemos el Tablero de la Figura \ref{fig:tab8}, vemos que a un tablero de 4x4 vacío, con 4 caballos le basta para completarse, lo cual indica que sucede algo similar con el tablero de 3x3. Por esto diremos que si nuestro tablero es menor o igual a 4x4, no debemos ver aquellos casos que pongan más de 4 caballos, puesto que de seguro, eso no es una solución óptima. ¿Pero qué sucede con el resto de los casos?. Observemos lo siguiente, sea un tablero de nxm (no apto para este juego, pero sí para lo que se quiere mostrar) con n=5 y m$\geq$5. Tomemos por ejemplo un tablero de 5x5 vacío, pero con una fila de caballos en el centro (llamados c).


\begin{figure}[!htb]
\minipage{0.32\textwidth}
  \includegraphics[scale=1]{imagenes/tab10.png}
  \caption{Tablero 10}\label{fig:tab10}
\endminipage
\minipage{0.32\textwidth}
  \includegraphics[scale=1]{imagenes/tab11.png}
  \caption{Tablero 11}\label{fig:tab11}
\endminipage
\end{figure}

%\begin{figure}[htb]
%  \begin{center}
%      \includegraphics[scale=1]{imagenes/tab10.png}
%  \end{center}
%  \caption{Tablero 9}\label{fig:tab9}
%\end{figure}

%\begin{figure}[htb]
%  \begin{center}
%      \includegraphics[scale=1]{imagenes/tab11.png}
%  \end{center}
%  \caption{Tablero 10}\label{fig:tab10}
%\end{figure}

Notemos que en la Figura \ref{fig:diag} todas las casillas están amenazadas u ocupadas y sería un tablero completo. Es más, si sacáramos un solo caballo, el tablero dejaría de ser completo. Esto, si bien no tiene por que ser una solución óptima, es cuanto menos una solución que no usa una cantidad abusiva de caballos. ¿Qué sucedería si extendiéramos una columna más, como en Figura \ref{fig:tab10}?. Todas las casillas estarían ocupadas menos la de la fila del medio a la derecha. Osea que si la fila estuviera completa, sería un tablero completo. Nótese que si vuelvo a extender una columna sucede lo mismo. Volviendo otra vez a tableros válidos (o sea, de nxn), la idea recién mencionada nos hace pensar que, independientemente del tamaño del tablero, con una línea de caballos cada 5 filas, debería bastar para completar el tablero, y ver más caballos que eso sería un esfuerzo innecesario. Si el tablero no fuera múltiplo de 5, bastaría con poner 1 fila de caballos cada 5 filas normales, y de lo que me queda elijo una fila que sea céntrica respecto de las sobrantes, que son menores que 5, y la convierto en fila de caballos. Ver que si una fila céntrica completa 5 filas, en particular completa 4,3,2 y 1.

Por todo lo dicho anteriormente, si llamamos n a la cantidad de filas y columnas de un tablero, dado un tablero de nxn con n$<$5, tomamos como cota 4 caballos extras, y si n$\geq5$ tomo como cota $n*\left \lceil \dfrac{n}{5} \right \rceil$, que resultaría de meter una fila de caballos por cada 5 filas del tablero.

Sin embargo, agregar estas cotas, ¿puede producir que se obtenga un resultado no óptimo?. La pregunta casi que se contesta sola, pues las cotas justamente se basan en que conocemos una solución segura que, aunque no sabemos si es óptima, sí sabemoms que usa una determinada cantidad de caballos, y una solución que use más caballos que esta seguro no es óptima.

\begin{figure}
\begin{codebox}
\Procname{$\proc{El_señor_de_los_caballos}$} 
\li n $\leftarrow$ tamaño del tablero
\li k $\leftarrow$ cantidad de caballos preubicados
\li extras $\leftarrow$ 0
\li tab $\leftarrow$ tablero de  nxn ``vacío''
\li tab_final $\leftarrow$ tablero de nxn ``vacío'' donde irá el resultado
\li tab $\leftarrow$ setear ubicación de los caballos preubicados y las casillas que amenaza cada uno.
\li falta_cubrir $\leftarrow$ cantidad de casillas no ocupadas 
\li \If ($n < 5$)
\li \Then cota $\leftarrow$ 4
\li \Else cota $\leftarrow n*\left \lceil \dfrac{n}{5} \right \rceil$
	\End
\li {\sc Backtracking}(tab,tab_final,0,0,n)
\li Muestro por pantalla: cantidad de caballos agregados y la posición de cada caballo extra
\end{codebox}
\caption{Pseudocódigo de El señor de los caballos}\label{code:caballos}
\end{figure}
\FloatBarrier

\begin{figure}[!ht]
\begin{codebox}
\Procname{$\proc{Backtracking}(Tablero$ $tab, Tablero$ $tab$_$final, int$ $fila, int$ $columna, int$ $n)$}
\li \If agregué más caballos de los que permitía la cota
\li \Then \Return   
	\End
\li \If ya llené el tablero
\li \Then tab_final $\leftarrow$ copia de tab
\li 		 cota $\leftarrow$ cantidad de caballos agregados 
\li 		\Return
	\End
\li \If ya recorrí todo el tablero
\li \Then \Return
	\End
\li f $\leftarrow$ fila para la siguiente llamada recursiva
\li c $\leftarrow$ columna para la siguiente llamada recursiva 
\li \If en tab[fila][columna] no había un caballo preubicado
\li \Then copia_tab $\leftarrow$ copia de tab
\li 		 r $\leftarrow$ cantidad de casillas que falta llenar
\li		 e $\leftarrow$ cantidad de caballos agregados hasta ahora
\li 		\If copia_tab[fila][columna] no está cubierta
\li 		\Then decrementar r
		\End
\li 		copia_tab[fila][columna] $\leftarrow$ coloco un caballo extra en esa posición
\li		copia_tab $\leftarrow$ seteo las casillas que amenaza ese nuevo caballo extra
\li 		incrementar e
\li 		{\sc Backtraking}(copia_tab,tab_final,f,c,n)
		\End
\li {\sc Backtraking}(tab,tab_final,f,c,n)
\end{codebox}
\caption{Pseudocódigo del backtraking}\label{code:backtraking}
\end{figure}
%\FloatBarrier

%\textbf{completar!}

%\begin{codebox}
%\Procname{$\proc{ejemplo_de_pseudocodigo}(x,y)$}
%\li \Return $\id{solucion}$
%\end{codebox}

\vspace*{0.6cm}

%\newpage
%\subsection{Justificación de la resolución y demostración de correctitud.}

%\vspace*{0.3cm}

%\textbf{completar!}

%\vspace*{0.6cm}

%\newpage
\subsection{Análisis de complejidad.}

\vspace*{0.3cm}

Ahora procederemos a demostrar que el algoritmo que planteamos posee una complejidad de $\mathcal{O}(n^2*2^{n^2})$. Luego de un simple vistazo al pseudocodigo (Figura \ref{code:caballos}) se puede ver que la complejidad deriva en gran medida de lo que sucede cuando se llama a la función encargada del backtracking, dado que las comparaciones realizadas son $\mathcal{O}(1)$ asi como las asignaciones. La complejidad de copiar la entrada y generar el tablero adecuado nos resulta $\mathcal{O}(n^2)$.

Analicemos ahora un caso particular, como el de un tablero de 2x2 sin caballos preubicados. Haciendo un leve seguimiento del pseudocodigo (Figura \ref{code:backtraking}) nos encontramos revisando la primer casilla del tablero y dado que no hay un caballo preubicado se generan 2 instancias, un nuevo tablero con un caballo extra en esa posicion y otro en el que este no se encuentra presente. La creación de este nuevo tablero nos cuesta $\mathcal{O}(n^2)$. Sobre estas dos instancias se vuelve a llamar a la función de backtracking y el proceso se repite ahora con la casilla siguiente, y asi hasta cubrir las $n*n$ casillas. Al terminar el recorrido notamos que contamos con 15 instancias ($2^{n^2} - 1$) y que por cada una tenemos un costo de $\mathcal{O}(n^2)$.
Este proceso de recorrer todas las casillas se repite para cualquier tablero vacío, creando $2^n -1$ instancias y llegando a una complejidad de $\mathcal{O}(n^2*2^{n^2})$ (el -1 es despreciable).

Ahora lo que sucede cuando contamos con caballos preubicados en un tablero cualquiera es que la comparación realizada en la linea 12 de Figura Backtracking, nos ahorra la creacion y copia del nuevo tablero, restando un $\mathcal{O}(n^2)$ a nuestra complejidad. A su vez, la cantidad de ramas con posibles soluciones se ven reducidas considerablemente resultando en $2^{n^2-k}$ instancias.

En conclusión nuestro algoritmo, en el peor caso, dado que no hay caballos prefijados, tiene la complejidad de $\mathcal{O}(n^2*2^{n^2})$.

\vspace*{0.6cm}

%\newpage
\subsection{Experimentación y gráficos.}

\vspace*{0.3cm}

En esta sección normalmente trataríamos de analizar empíricamente que la complejidad de nuestro algoritmo es exponencial. Sin embargo, si realmente fuera exponencial, entonces hacer un análisis de complejidad sería muy complicado por un tema de tiempos. Es por esto, que los experimentos realizados, mostrarán cómo al aumentar el valor de n, los tiempos cambian drásticamente. Por otro lado, veremos cómo el agregar caballos preubicados disminuye en gran proporción el tiempo utilizado por nuestro algoritmo, y qué ocurre con el caso de un tablero lleno.

\subsubsection{Test 1}

\vspace*{0.3cm}

Como primera medida, creamos instancias de tableros de 2x2, 3x3, 4x4 y 5x5, todos vacíos, es decir, sin caballos preubicados.  Cada instancia se ejecutó 50 veces y se consideró el promedio de los tiempos incurridos para cada caso (Figura \ref{fig:3-vacios}).

\begin{figure}[htb]
	\begin{center}
    		\includegraphics[scale=0.5]{imagenes/3-vacios.jpg}
	\end{center}
	\caption{El Señor de los Caballos - Tableros vacíos}\label{fig:3-vacios}
\end{figure}

Observemos que en el gráfico de la Figura \ref{fig:3-vacios} el crecimiento no es lineal y que n=2 es más de 1800 veces más rápido que n=5. De hecho, observemos que en el gráfico, el eje 'y' está en escala logarítmica, y aún así, si unimos los puntos con una línea, da algo mayor que lineal. Esto es un indicio de que nuestro algoritmo es exponencial, lo cual apoya nuestro analisis de complejidad.

\vspace*{0.6cm}

%\newpage
\subsubsection{Test 2}

\vspace*{0.3cm}

En este caso, repetimos la experimentación anterior, pero con los tableros llenos de caballos preubicados (Figura \ref{fig:3-llenos}).

\begin{figure}[htb]
	\begin{center}
    		\includegraphics[scale=0.5]{imagenes/3-llenos.jpg}
	\end{center}
	\caption{El Señor de los Caballos - Tableros vacíos}\label{fig:3-llenos}
\end{figure}

Veamos que la diferencia de tiempos es realmente mínima, y el tiempo de ejecución es realmente pequeño, con un pico en 5. Esto probalemente se deba a que, por como está hecho nuestro algoritmo, si el tablero está lleno, nos ahorramos el costo de recorrerlo y nos limitamos a devolver el tablero de entrada, ya que este es solución. Las diferencias de tiempo (sobre todo la vista en n=5), puede explicarse por el costo de copiar el tablero inicial como tablero solución, que es $\mathcal{O}(n^2)$.

\vspace*{0.6cm}

%\newpage
\subsubsection{Test 3}

\vspace*{0.3cm}

Para este caso, decidimos crear 9 instancias de tableros de 5x5.  Para cada uno, fueron insertándose caballos preubicados de manera aleatoria.  El tablero con 2 caballos preubicados, por ejemplo, tiene el mismo caballo preubicado que el tablero con 1 caballo, y uno más.  Los 9 tableros contemplan desde 0 hasta 8 caballos preubicados.

Nuevamente, para cada instancia se ejecutó el programa 50 veces y se consideró el promedio de los tiempos incurridos (Figura \ref{fig:3-random}).

\begin{figure}[htb]
	\begin{center}
    		\includegraphics[scale=0.5]{imagenes/3-random.jpg}
	\end{center}
	\caption{El Señor de los Caballos - Tableros vacíos}\label{fig:3-random}
\end{figure}

Observamos que hay una abrupta disminución de tiempo por cada caballo agregado. Esto se podría deber a que cuando hay un caballo preubicado, nuestro algoritmo no debe decidir entre poner un caballo o no ponerlo, y se ahorra toda la rama de decisión de poner un caballo. Esto último ya fue abordado en el apartado de análisis de complejidad.

En conclusión, si bien no podemos decir que nuestro algoritmo es exponencial de manera empírica, las dificultades generadas por los tiempos de experimentación y los experimentos, apoyan la teoría de que nuestro algoritmo pueda ser del orden exponencial.
\subsection{Descripción del problema.}

\vspace*{0.3cm}

\textbf{Tenemos un juego de ajedrez, que solo utiliza caballos, y el tablero mide nxn donde n es varible, el juego es solamente de un jugador.
Inicialmente tenemos k caballos ubicados en cualquiera de las casillas de ajedrez, estos caballos deben quedar fijos.
Podemos agragar caballos al tablero de manera de ocupar todos los casilleros, un tablero se considera ocupado
si tiene un caballos o si esta siendo  amenazado por algun otro caballo.
La idea del juego es utilizar la menor cantidad de caballos posibles, de manera de ocupar todas las casillas.
}


%\begin{figure}[htb]
%  \begin{center}
%      \includegraphics[scale=0.25]{imagenes/ejemplo.jpg}
%  \end{center}
%  \caption{ejemplo}
%\end{figure}



\newpage
\subsection{Desarrollo de la idea y pseudocódigo.}

\vspace*{0.3cm}

\textbf{completar!}

\begin{codebox}
\Procname{$\proc{ejemplo_de_pseudocodigo}(x,y)$}
\li \Return $\id{solucion}$
\end{codebox}



\newpage
\subsection{Justificación de la resolución y demostración de correctitud.}

\vspace*{0.3cm}

\textbf{completar!}



\newpage
\subsection{Análisis de complejidad.}

\vspace*{0.3cm}

\textbf{completar!}



\newpage
\subsection{Experimentación y gráficos.}

\vspace*{0.3cm}

\subsubsection{Test 1}

\vspace*{0.3cm}

\textbf{completar!}


\newpage
\subsubsection{Test 2}

\vspace*{0.3cm}

\textbf{completar!}


\newpage
\subsubsection{Test 3}

\vspace*{0.3cm}

\textbf{completar!}

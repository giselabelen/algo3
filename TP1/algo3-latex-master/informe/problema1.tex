\subsection{Descripción del problema.}

\vspace*{0.3cm}

\textbf{
Tenemos un pais infectado de zombis. Nuestro ojetivo es salvar la mayor cantidad de ciudades de este pais.
Cosas que tenemos que tener en cuenta:
}
\begin{itemize}
   \item Los recursos del pais son acotados 
   \item Cada ciudad tiene una cantidad de soldados atrincherados(Soldados iniciales de la ciudad)  
   \item cada ciudad tiene una cantidad de zombis  
   \item cada ciudad tiene un costo de envio por soldado
   \item cada soldado puede acabar como maximo con 10 zombis 
\end{itemize}

\textbf{
La manera de salvar una cuidad es la siguiente:
}         
\begin{itemize}
	\item El pais puede mandar soldados de refuerzo a cada ciudad, de manera que ayuden a los soldados atrincherados
	\item Una ciudad se cosidera salvada si los soldados acaban con todos los zombies de dicha ciudad
\end{itemize}


%\begin{figure}[htb]
%  \begin{center}
%      \includegraphics[scale=0.25]{imagenes/ejemplo.jpg}
%  \end{center}
%  \caption{ejemplo}
%\end{figure}



\newpage
\subsection{Desarrollo de la idea y pseudocódigo.}

\vspace*{0.3cm}

\textbf{
Tomamos de entrada la el estado actual de pais, donde estado actual indica:
}
\begin{itemize}
   \item  n que indica las la cantidad total de cuidades 
   \item p el presupuesto del pais  
   \item a el arreglo de tamaño n, donde el contenido del arreglo es:
   \begin{itemize}
		\item nombre: que indica el numero de ciudad
		\item cant_zombies : que indica la cantidad de zombis
		\item cant_soldados : que indica la cantidad de soldados
		\item cant_soldados : que indica la cantidad de soldados
		\item costo_por_soldado : costo de envio de un soldado
   \end{itemize}  
\end{itemize}
\textbf{
la idea es calculando el costo de salvacion de cada ciudad, donde el costo de salvacion calcula 
cuanto me va salir salvar la ciudad:   
}
\begin{itemize}
   \item  zombis_vivos = zombis_totales - 10 * soldados_actuales
   \item  soldados_requieridos = zombis_vivos/ 10
   \item costo_salvacion = soldados_requeridos * costo_por_soldado  
\end{itemize}
\textbf{
Es costo de salvacion de cada ciudad va a tomar lineal n tiempo. Luego ordeno las ciudades de menor a mayor
por el costo_salvacion(usando un mergesort). Ahora busco la solucion goloza, para cada ciudad:
}
\begin{itemize}
	\item voy sumando la costo de salvacion de la ciudad i con la ciudad i+1, hasta que la suma 
	supere el presupuesto o las ciudades se acaben, 
	\item luego retorno la cantidad de ciudades que puedo salvar
\end{itemize}
\textbf{
Por ultimo vuelvo a ordenar las ciudades por nombre de ciudad usando un merge sort 
}

\begin{codebox}
\Procname{$\proc{ejemplo_de_pseudocodigo}(x,y)$}
\li \Return $\id{solucion}$
\end{codebox}



\newpage
\subsection{Justificación de la resolución y demostración de correctitud.}

\vspace*{0.3cm}

\textbf{completar!}



\newpage
\subsection{Análisis de complejidad.}

\vspace*{0.3cm}

\textbf{completar!}



\newpage
\subsection{Experimentación y gráficos.}

\vspace*{0.3cm}

\subsubsection{Test 1}

\vspace*{0.3cm}

\textbf{completar!}


\newpage
\subsubsection{Test 2}

\vspace*{0.3cm}

\textbf{completar!}


\newpage
\subsubsection{Test 3}

\vspace*{0.3cm}

\textbf{completar!}

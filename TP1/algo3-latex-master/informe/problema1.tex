\subsection{Descripción del problema.}

\vspace*{0.3cm}

\textbf{
Tenemos un pais infectado de zombis. Nuestro ojetivo es salvar la mayor cantidad de ciudades de este pais.
Cosas que tenemos que tener en cuenta:
}
\begin{itemize}
   \item Los recursos del pais son acotados 
   \item Cada ciudad tiene una cantidad de soldados atrincherados(Soldados iniciales de la ciudad)  
   \item cada ciudad tiene una cantidad de zombis  
   \item cada ciudad tiene un costo de envio por soldado
   \item cada soldado puede acabar como maximo con 10 zombis 
\end{itemize}

\textbf{
La manera de salvar una cuidad es la siguiente:
}         
\begin{itemize}
	\item El pais puede mandar soldados de refuerzo a cada ciudad, de manera que ayuden a los soldados atrincherados
	\item Una ciudad se cosidera salvada si los soldados acaban con todos los zombies de dicha ciudad
\end{itemize}


%\begin{figure}[htb]
%  \begin{center}
%      \includegraphics[scale=0.25]{imagenes/ejemplo.jpg}
%  \end{center}
%  \caption{ejemplo}
%\end{figure}



\newpage
\subsection{Desarrollo de la idea y pseudocódigo.}

\vspace*{0.3cm}

\textbf{completar!}

\begin{codebox}
\Procname{$\proc{ejemplo_de_pseudocodigo}(x,y)$}
\li \Return $\id{solucion}$
\end{codebox}



\newpage
\subsection{Justificación de la resolución y demostración de correctitud.}

\vspace*{0.3cm}

\textbf{completar!}



\newpage
\subsection{Análisis de complejidad.}

\vspace*{0.3cm}

\textbf{completar!}



\newpage
\subsection{Experimentación y gráficos.}

\vspace*{0.3cm}

\subsubsection{Test 1}

\vspace*{0.3cm}

\textbf{completar!}


\newpage
\subsubsection{Test 2}

\vspace*{0.3cm}

\textbf{completar!}


\newpage
\subsubsection{Test 3}

\vspace*{0.3cm}

\textbf{completar!}

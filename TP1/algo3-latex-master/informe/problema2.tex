\subsection{Descripción del problema.}

\vspace*{0.3cm}

\textbf{
Tenemos un enlace por el cual podemos transmitir información, utilizando distintas frecuencias.
Nuestro objetivo es transmitir información durante todo el tiempo que sea posible, pero invirtiendo la menor cantidad de dinero posible.
\\
Hay que tener en cuenta lo siguiente:
}
\begin{itemize}
	\item vamos a transmitir información secuencialmente, es decir solamente se utilizara una frecuencia por intervalo tiempo 
	\item cada frecuencia tiene un intervalos $<$inicio, fin$>$ en el cual puede ser utilizado y un costo por minuto de uso.Tanto inicio como fin son numeros enteros que representan minutos. 
	\item El enlace puede eligir transmitir por cualquiera de las frecuencias disponibles, y cambiar de una frecuencia a otra.
	\item Se puede cambiar la frecuencia usada a lo largo del tiempo, pero solo una vez por minuto y al comienzo del mismo.
\end{itemize}

\textbf{Nuestro algoritmo toma:}
\begin{itemize}
	\item n que indica la cantidad frecuencias que puede utilizar el enlace
	\item tenemos n lineas seguidas, cada linea indica una frecuencia
	\begin{itemize}
		\item cada linea tiene campos <c,i,f> que indican costo por minuto, inicio y fin del intervalo de disponibilidad
	\end{itemize}
	\item El tiempo inicial empieza a partir del minuto cero
\end{itemize}

\textbf{Salida:}
\begin{itemize}
	  \item costo total incurrido de la solucion( osea minimizando los costos)
	  \item una linea por cada intervalo de transmicion 
\end{itemize}

%\begin{figure}[htb]
%  \begin{center}
%      \includegraphics[scale=0.25]{imagenes/ejemplo.jpg}
%  \end{center}
%  \caption{ejemplo}
%\end{figure}



\newpage
\subsection{Desarrollo de la idea y pseudocódigo.}

\vspace*{0.3cm}

\textbf{completar!}

\newpage
\subsection{Justificación de la resolución y demostración de correctitud.}

\vspace*{0.3cm}

\textbf{completar!}



\newpage
\subsection{Análisis de complejidad.}

\vspace*{0.3cm}

\textbf{completar!}

\begin{codebox}
\Procname{$\proc{AltaFrecuencia}()$} 
\li Creo una lista de transmisiones resultado  //$\mathcal{O}(1)$ 
\li total_freq $\leftarrow$ Creo un arreglo para almacenar las n frecuencias levantadas de la entrada $\mathcal{O}(n)$
\li ordeno_por_tiempo(total_freq,0,n-1) {\it //ordeno por tiempo de inicio. $\mathcal{O}(n.logn)$}
\li resultado $\leftarrow$ frecuency_dc(total_freq,0,n-1)
\li {\it //armo la transmisión resultado haciendo uso de la técnica de D\&C}
\li costo_total $\leftarrow$ costo_transmision(resultado) //$\mathcal{O}(n)$
\li Muestro por pantalla: costo de la transmisión (costo_total) y el intervalo de tiempo ocupado por cada\\ frecuencia usada {\it //Se hace en $\mathcal{O}(n)$ porque es recorrer la lista mostrando las frecuencias $\mathcal{O}(1)$}
\end{codebox}

\begin{codebox}
\Procname{lista_de_transmisión $\proc{frecuency_dc}(frecuencia$*$ freq, int$ $low, int$ $high)$}
\li Inicializo 3 listas: trans1,trans2,trans_final
\li \If ($low<high$)
\li	\quad	mitad $\longleftarrow$ $\frac{low + high}{2}$
\li	\quad	trans1 $\longleftarrow$ $frequency$_$dc(freq,low,mitad)$
\li	\quad	trans2 $\longleftarrow$ $frequency$_$dc(freq,mitad+1,high)$
\li	\quad	A continuación combino ambas listas de transmisiones de manera que queden ordenadas por costo
\\ \quad de menor a mayor y que las frecuencias alternen de ser necesario.
\li \Else
\li \quad	Armo una lista cuyo único elemento es la frecuencia de la posicion "low". Libero la memoria ocupada\\ \quad por trans1 y trans2
\li \Return trans_final 
 \end{codebox}


\newpage
\subsection{Experimentación y gráficos.}

\vspace*{0.3cm}

\subsubsection{Test 1}

\vspace*{0.3cm}

\textbf{completar!}


\newpage
\subsubsection{Test 2}

\vspace*{0.3cm}

\textbf{completar!}


\newpage
\subsubsection{Test 3}

\vspace*{0.3cm}

\textbf{completar!}
